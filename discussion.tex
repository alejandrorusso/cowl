\section{Discussion and Limitations}
\label{sec:discussion}

We now discuss the implications of certain facets of \sys{}'s
design, and limitations of the system.

%% In this section we discuss various design trade-offs, the limitations
%% of \sys{} that arise as result of these trade-offs, and future work
%% directions for addressing some of these limitations.

\paragraph{User-Configured Confinement}
%%Recall that in the status-quo web security architecture, the SOP by
%%default blocks an origin from reading the response from a server of a
%%different origin. To allow cross-origin sharing, a server can
Recall that in the status-quo web security architecture,
to allow cross-origin sharing, a server must
grant individual foreign origins access to its data with CORS in
all-or-nothing, DAC fashion.
%
COWL improves this state of affairs by allowing a COWL-aware server to
more finely restrict how its shared data is
disseminated---\emph{i.e.,} when the server grants a foreign origin
access to its data, it can confine the foreign origin's script(s) by
setting a label on responses it sends the client.

Unfortunately, absent a permissive CORS header that whitelists the
origins of applications that a user wishes to use, the SOP
prohibits foreign origins from reading responses from the server,
even in a COWL-enabled browser.
%
Since a server's operator may not be aware of all applications its
users may wish to use, the result is usually the same status-quo
unpalatable choice between functionality and privacy---\emph{e.g.,} give
one's bank login credentials to Mint, or one cannot use the Mint
application.
%
For this reason, our COWL implementation lets browser users augment
CORS by configuring for an origin (\emph{e.g.,} \https{chase.com}) any
foreign origins (\emph{e.g.,} \https{mint.com}, \https{benjamins.biz})
they wish to additionally whitelist.
%
In turn, COWL will confine these client-whitelisted origins (\emph{e.g.,}
\https{mint.com}) by labeling every response from the configured
origin (\https{chase.com}).
%
COWL obeys the server-supplied label when available and server
whitelisting is \emph{not} provided. Otherwise, COWL conservatively
labels the response with a \emph{fresh} origin (as described in
Section~\ref{sec:system:privileges}). The latter ensures that once the
response has been inspected, the code cannot communicate with
\emph{any} server, including at the \emph{same} origin, since such
requests carry the risks of self-exfiltration~\cite{selfex} and
cross-site request forgery~\cite{CSRF}.
%


\paragraph{Covert Channels}
In an ideal confinement system, it would always be safe to let untrusted
code compute on sensitive data.
%
Unfortunately, real-world systems such as browsers typically exhibit
\emph{covert} channels that malicious code may exploit to exfiltrate
sensitive data.
%
Since COWL extends existing browsers, we do not protect against covert
channel attacks.
%
Indeed, malicious code can leverage covert channels already present in
today's browsers to leak sensitive information.
%
For instance, a malicious script within a confined context may be able
to modulate sensitive data by varying rendering durations. A less
confined context may then in turn exfiltrate the data to a remote
host~\cite{kotcher2013cross}.
%
It is important to note, however, that COWL does not introduce new
covert channels---our implementations re-purpose existing
(software-based) browser isolation mechanisms (V8 contexts and
SpiderMonkey compartments) to enforce MAC policies.
%
Moreover, these MAC policies are generally more restricting than
existing browser policies: they prevent unauthorized data exfiltration
through \emph{overt} channels and, in effect, force malicious code to
resort to using covert channels.


The only fashion in which COWL relaxes status-quo browser policies is
by allowing users to override CORS to permit cross-origin (labeled)
sharing.
%
Does this functionality introduce new risks?
%
Whitelisting is user controlled (\emph{e.g.,} the user must explicitly
allow \https{mint.com} to read \https{amazon.com} and
\https{chase.com} data), and code reading cross-origin data is subject
to MAC (\emph{e.g.,} \https{mint.com} cannot arbitrarily exfiltrate
the \https{amazon.com} or \https{chase.com} data after reading it).
%
In contract, today's mashups like \https{mint.com} ask users for their
passwords.
%
COWL is strictly an improvement: under
COWL, when a user decides to trust a mashup integrator such as
\https{mint.com}, she {\em only} trusts the app to not leak her data
through covert channels.
%
Nevertheless, users can make poor security choices. Whitelisting
malicious origins would be no exception; we recognize this as a
limitation of COWL that must be communicated to the
end-user.% (much like accepting SSL certificates).

A trustworthy developer can leverage COWL's support for
\emph{clearance} when compartmentalizing his application to ensure
that only code that actually relies on cross-origin data has access to
it.
%
Clearance is a label that serves as an upper bound on a context's
current label. Since COWL ensures that the current label is adjusted
according to the sensitivity of the data being read, code cannot read
(and thus leak) data labeled above the clearance.
%
Thus, Mint can assign a ``low'' clearance to untrusted third-party
libraries, {\em e.g.,} to keep \https{chase.com}'s data confidential.
%
These libraries will then not be able to leak such data through covert
channels, even if they are malicious.

%%%%

%
% \todo{}{it's important for us to note that we do not address covert
% channels even for conservative-design, since developers may have the
% impression that they're getting bullet-proof security when sending a
% password to another frame, when in actuality it can be leaked (though
% at a low bit rate)}



\paragraph{Expressivity of Label Model}
\label{sec:discussion:lattice}

COWL uses DC labels~\cite{stefan:2011:dclabels} to enforce
confinement according to an information flow control discipline.
%
Although this approach captures a wide set of confinement policies, it
is not expressive enough to handle policies with a circular flow of
information~\cite{Badger:1995} or some policies expressible in more
powerful logics (\emph{e.g.,} first order logic, as used by
Nexus~\cite{sirer2011logical}).
%
%% Indeed, even allowing negative formulas in a label would permit more
%% expressive policies (\emph{e.g.,} ``this labeled Blob cannot be read
%% by \https{a.com}, but can be read by any other origin''); with DC
%% labels, such policies must be encoded by `whitelisting'' all permitted
%% readers.
%
DC labels are, however, as expressive as other popular label
models~\cite{GenLabels}, including Myers and Liskov's Decentralized
Label Model~\cite{dlm}. Our experience implementing security policies
with them thus far suggests they are expressive enough to support
featureful web applications.

We adopted DC labels largely because their fit with web origins pays
practical dividends.
%
%That correspondence yields two important benefits.
%
First, as developers already typically express policies by whitelisting
origins, we believe they will find DC labels intuitive to use.
%and expressive enough to build real applications.
%
%% (Recall that developers typically express policies by whitelisting
%% origins.)
%
%% Our experience building applications, some described in this paper,
%% supports this hypothesis. Their use in server-side applications is
%% also encouraging~\cite{giffin:2012:hails, Breeze13, stoughton2014you,
%%   stefan:2011:flexible}.
%
Second, because both DC labels and today's web policies are defined in
terms of origins, the implementation of COWL can straightforwardly
reuse the implementation of existing security mechanisms, such as CSP.
%% %
%% (This was a surprising given our previous work showed how IFC can be
%% used to enforce existing policies, but not the
%% converse~\cite{yang:2013:towards}.)


% Local Variables:
% TeX-master: "main.ltx"
% TeX-command-default: "Make"
% End:

