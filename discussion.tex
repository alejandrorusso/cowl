\section{Discussion and Limitations}
\label{sec:discussion}


\subsection{Covert channels}

%% Intro covert channel
As many confinement systems, \sys{} is not flawless. The existence of covert
channels allows malicious scripts to reveal sensitive
information~\cite{lampson:confinement}.  Such channels often arise when
programming language features are misused to leak information; The tolerance to
their presence is usually determined by their bandwidth and how easy is to
exploit them.

%% Browbound suffer from the termination channel
\sys{} suffers from the termination covert channel. Scripts are then capable to
reveal information by diverging based on confidential values. The reason for
that relies in the implementation of labeled workers: \emph{they all run in the
  same thread}. If a labeled worker diverges, then the corresponding browser tab
hangs. The best strategy for malicious scripts is then to perform a brute-force
attack every time that a page is loaded, e.g., a worker hangs when guessing the
exact value of the secret~\cite{askarov:termination-insensitive}. In our
setting, the termination channel has a capacity where the time needed to reliably
leak the secret is exponential in the number of secret bits.

%% Solution: labeled workers as real workers (threads)

%% a) Labeled workers as workers
%% b) We have a limited of 5 
%% c) Useless, since we do not have restriction how to create threads
%% d) Having them in a different threads will require for us to instrument some 
%%    synchronizaiton to accesing the DOM

At first glance, labeled workers could be run on separate threads, which indeed
removes the termination channel~\cite{stefan:addressing-covert}.  Unfortunately,
this idea might not be feasible to implement in modern web browsers. Firefox and
Chrome bound the number of workers (running on separate threads) to five. The
reason for that being two fold; first, the creation of threads is heavy-weight,
and second, workers are not intended  to be used in large numbers
by web pages. 
%\sys{} could remove the termination channel at the price of
%allowing a maximum of five labeled workers which are guarantee to terminate.  
Observe that the termination channel could still be exploited.  It would be
enough for a script to exhaust the number of allowed labeled workers (five) in
order to make a program crash.
%\sys{} could remove the termination channel at the price of
%allowing a maximum of five labeled workers. 
Alternatively, and as it is now, \sys{}
permits to freely create any number of labeled workers, while leaving the
termination channel exploitable only by brute-force attacks.

% \Red{Make sure to note that in the presence of covert channels
% relaxing the SOP is very dangerous. So, COLWorkers could really be
% limited to extensions/pages that explicitly request this feature.}

\subsection{DC labels}

The implementation of \sys{} is tighten to DC labels. This design decision has
several consequences. Firstly, users familiar with other labeling systems
(e.g. DLM) might need to perform some encoding in order to use
them~\cite{GenLabels}.  Secondly, and more importantly, \sys{} only enforces
latticed-based policies, i.e., labels dictate where information can be
disseminated.  While useful for our purposes, such policies might not capture
all the desire ones for confinement. For instance, it is known that
lattice-based policies are not suitable for scenarios with circular flows of
information~\cite{Badger:1995}. Moreover, DCS~\cite{Akhawe2013} enforces any
confinement policy which can be written as a program. While DCS support more
powerful policies than \sys{}, writing policies as programs might be error
prone. Instead, we design \sys{} to enforce weaker, although useful,
confinement policies based on the specification of data labels. 

\subsection{Granularity of enforcement } 

%% DCS writing a policy as a program. 
%% Circular policies. 
%% 

% Can only express latticed-based policies. Can't express all kinds of
% policies. But we speculate that this is good enough; i.e., dclabels
% are a good match for the web scenario where we already express
% policies in terms of origins.



% Local Variables:
% TeX-master: "main.ltx"
% TeX-command-default: "Make"
% End:

