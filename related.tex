\section{Related Work}
\label{sec:related}

Existing implementations of information flow control in the browser can
be classified either as \emph{fine-grained}, which associate IFC
policies with individual objects in the browser, or
\emph{coarse-grained}, which associate policies with
entire compartments and browsing contexts.  Because \sys{} is a
coarse-grained IFC system, we first compare it to other coarse-grained
systems, and then talk both generally and in particular with how
it compares to fine-grained systems.

\paragraph{Coarse-grained IFC} \sys{} shares many features
with coarse-grained systems.
% BFlow
%
BFlow~\cite{Yip:2009:PBS}, for example, allows web sites to enforce confinement policies
stricter than the SOP\@.
%
BFlow can confine third-party code by placing
it in a tainted \emph{protection zone}---a group
of iframes that share a common label.
%
However, unlike \sys{}, BFlow cannot mediate between mutually
distrustful principals: e.g., one cannot directly implement the
encrypted document editor with BFlow.
%
This is because BFlow does not provide symmetric confinement---a
sub-frame cannot impose any restrictions on its parent---and its
labels do not support conjunctions of multiple origins (e.g., BFlow
cannot encode a policy that states that sensitive data may derive from
data from two web sites).\footnote{ The mutually distrusting scenario
  is further complicated by BFlow's reliance on top-level
  \emph{trusted zones} that can always \emph{completely} declassify
  the data in their corresponding page.}
%
For the same reasons, BFlow cannot support applications that require
security policies more flexible than the SOP, such as our third-party
mashup example.
%
These differences reflect different goals for the two systems: BFlow's
authors set out to confine untrusted third-party scripts; in contrast,
we seek to support applications that incorporate code from mutually
distrusting parties.


% DCS 
More recently, Akhawe \emph{et al.}~proposed the data-confined
sandbox (DCS) system~\cite{Akhawe2013}, which provides pages with the
ability to intercept and monitor the network, storage, and cross-origin
channels of certain iframes.
%
Such monitoring can in principle confine code such as the
password-strength checker.
%
However, the way DCS implements confinement is fundamentally limited to \verb|data:|
URI iframes, and, as a result, DCS cannot confine a service provided
in an iframe (the common case, both to isolate the enclosing page from
the script and because the checker's author may not want to share
source code\cite{postman}).
%
Like BFlow, DCS does not offer symmetrical confinement---an iframe
cannot impose restrictions on its parent. It therefore cannot support
applications that combine code from mutually distrusting origins, as
does the encrypted document editor example.
%
DCS further does not incorporate functionality to let developers build
applications like third-party mashups, which require safe relaxation of the
SOP\@.

\paragraph{Fine-grained IFC}
Fine-grained IFC systems differ from \sys{} in that they track
information flow at the level of objects, allowing applications
to take advantage of information flow control without explicitly
compartmentalizing their applications. This makes it easier to
confine untrusted libraries which are closely coupled with trusted
code on a page (as is jQuery).
% JSFlow 
JSFlow~\cite{JSFlow} is one such fine-grained JavaScript IFC system, which
enforces policies by executing JavaScript in an interpreter written in
JavaScript.
%
One difficulty with JSFlow is that
its implementation incurs a two orders of magnitude slowdown:
JSFlow's authors suggest that this cost makes JSFlow a better fit for use as a development tool than
as an ``always-on'' privacy system for users' browsers.
%
Additionally, JSFlow does not support applications that rely on policies
more flexible than the SOP, such as our third-party mashup example.

% With our
Another fine-grained IFC system is FlowFox~\cite{DeGroef:2012},
which uses secure-multi execution
(SME)~\cite{Devriese:2010} in order to enforce policies. SME ensures that no leaks
from a sensitive context can leak into a less sensitive context by
executing a program multiple times.
%
Unlike JSFlow and \sys{}, SME is not amenable to
scenarios where declassification plays a key role (e.g., the encrypted
editor, password manager, or third-party mashup).
%
FlowFox's labeling of user interactions and metadata (history, screen
size, etc.) do allow it to mitigate history sniffing and behavior
tracking;
these attacks are not addressed by \sys{}, where developers must
impose policy on the data to be shared.

Ultimately, however, coarse- and fine-grained mechanisms are not mutually
exclusive; for instance, when dealing with legacy JavaScript code, JSFlow could be
implemented within a COWL compartment.

\paragraph{Sandboxing}
There is much work on sandboxing and developing subsets of JavaScript (e.g.,
Caja~\cite{GoogleCaja}, BrowserShield~\cite{Reis:2007},
WebJail~\cite{VanAcker:2011}, TreeHouse~\cite{Ingram:2012},
JSand~\cite{Agten:2012:JCC}, SafeScript~\cite{SafeScript}, Defensive
JavaScript~\cite{djs}, and Embassies~\cite{Howell:2013}). 
%
While our design has been inspired by some of these systems (e.g.,
TreeHouse), the usual goals of these systems are to mediate
security-critical operations, restrict access to the DOM, and restrict
communication APIs\@.
%
In contrast to the mandatory nature of confinement, however, most restrictions
are imposed in a discretionary fashion and are thus not suitable to the
building a number of the applications we consider (in particular, the encrypted
editor).
%
Nevertheless, we believe that access control and language subsets are a crucial
complement to confinement when building robust secure applications.

% Local Variables:
% TeX-master: "main.ltx"
% TeX-command-master: "make"
% tex-dvi-view-command: "gmake preview;:"
% End:
