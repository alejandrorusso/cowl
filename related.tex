\section{Related Work}
\label{sec:related}

% BFlow
\Red{BFlow}
The closest related work is BFlow~\cite{Yip:2009:PBS}.
%
Like \sys{}, BFlow allows websites to enforce confinement policies
that are stricter than the SOP.
%
BFlow can, for example, be used to confine third-party code such as
the password strength checker by placing the untrusted code in a
tainted \emph{protection zone}, which is a group of iframes that share
a common label.
%
However, unlike \sys{}, BFlow is not amendable to a mutually
distrusting scenario and thus new applications such as the encrypted
document editor and password manager cannot be implemented without
giving up trust.
%
This stems from two design design points.
%
First, BFlow uses hierarchical confinement: the label of a subframe
(of a different protection zone) must have a more restricting label
than its parent; hence, an frame cannot impose any restrictions on its
parent.
%
Second, BFlow labels cannot encode the fact that data may be sensitive
to two websites and thus application, such as the encrypted editor and
password manager, that require one party to partially declassify a
piece of information cannot be expressed.
%
(For declassification, BFlow relies on a top-level \emph{trusted zone}
that can always \emph{completely} declassify the data in the page.)
%
This latter design point further prevents BFlow from being amendable
to building applications, such as our third-party mashup, that rely on
a policy more flexible than the SOP.
%
In fairness, BFlow did not consider applications with mutually
distrusting parties---their goal was to confine untrusted third-party
scripts, and indeed succeed at doing so.
%
Moreover, while \sys{} propagates labels using HTTP headers and is
amendable to confining one user's code from another on the same
website, we do not provide a server that implements this
functionality; developers can use BFlow's server-side component or
systems such as Hails~\cite{giffin:2012:hails} and
Aeolus~\cite{cheng:aeolus} to ensure that labels are propagated
appropriately in an end-to-end fashion.
%
\Red{I think the story might be even better for us: I don't think they
can build password checker where the latter is an iframe hosted on
other domain.}

%% BFlow~\cite{Yip:2009:PBS} is a confinement system for web browsers. It tracks
%% flows of information at the granularity of \emph{secure zones}, i.e.,
%% compartments composed of one or several iframes.  {\sys} does not require such
%% abstraction, it relies on the compartment notions already provided by browsers,
%% i.e., iframes and workers. Similar to our approach, BFlow uses \js|postMessage|
%% for communication across secure zones. BFlow does not allow JavaScript code in
%% different secure zones to write to shared DOM variables and cookies regardless
%% their security labels.  While \sys~ supports more expressive labels, potentially
%% including different domains, BFlow can propagates labeled data to the server
%% side. It is stated as future work to achieve a similar feature.


% DCS 
\Red{DCS}
More recently, Akhawe \emph{et al.} proposed extending the browser
with the data-confined sandboxe (DCS) system~\cite{Akhawe2013}.
%
DCS provides pages with the ability to intercept and monitor the
network, storage and cross-origin channels of confined iframes that
have \verb|data:| URI.
%
This, in turn, can be used used to confine code such as the password
strength checker.
%
Unfortunately, limiting the confinement to \verb|data:| URI iframes
means that DCS cannot be used to confine a password strength checker
service, i.e., code provided in the form of an iframe (but not
otherwise) by a third-party.
%
Moreover, DCS also lacks the symmetry property of \sys{}---an iframe
cannot impose restrictions on the parent---which is crucial to
building new applications such as the encrypted editor or password
manager.
%
Of course, this is not a 
%
Unlike \sys{}, DCS is not limited to using labels for policy
specification; instead, it allows a page to implement policies using
arbitrary JavaScript.
%
However, to ensure that the monitor cannot be used to learn more
information about the iframe content than allowed by the SOP, the
kinds of policies that a page can impose on an iframe is limited
(e.g., DCS cannot impose restrictions on redirects).
%
\sys{} does not have this limitation since the iframe itself imposes
the policy (albeit as a result of wishing to receive sensitive data
from the parent).
%
This design choice of using declarative policies further allows us to
relax the SOP and build applications such as the third-party mashup.
%
However, while we believe that using labels is sufficient for many
web applications, we recognize this as a limitations of \sys{} and
leave the investigation of expressiveness to future work.

%% Recently, and different from other sandboxes approaches, the
%% data-confined sandbox system~\cite{Akhawe2013} (DCS) restricts
%% propagation of information by using \js|iframes| and mediating
%% cross-domain operations (e.g. access to local storage, fragment-IDs,
%% network communication, etc.).  Every sandbox communicates by
%% \js!postMessage! only with its designated parent. The parent defines
%% the confinement policy and plays the role of man-in-the-middle when a
%% sandbox wish to communicate with another one. While having similar
%% goals, DCS is more restrictive than \sys. DCS disallows dynamic code
%% evaluation inside sandboxes and presents a not egalitarian design,
%% i.e., a sandbox (\js|iframe|) cannot host another sandbox. Clearly,
%% the DCS client-side TCB is smaller than ours.



% JSFlow 
JSFlow~\cite{JSFlow} is a fine-grained IFC system for JavaScript,
implemented in JavaScript.
%
By associates labels with individual objects, JSFlow can be used to
confine untrusted libraries, such as jQuery, that are intertwined with
the page page, which may need to communicate over the network.
%
\sys{} is coarse-grained and, like BFlow, generally requires the
application to be compartmentalized (e.g., by placing the page code
that needs to communicate with the network in a lightweight DOM
worker); 
%
Moreover, since JSFlow implements an interpreter they can associate
labels with, for instance, history data.
%
Unfortunately, the approach of providing an interpreter in JavaScript
for JavaScript has three main drawbacks.
%
First, JSFlow cannot be used to confine style sheets that execute code
or build applications that rely on policies more flexible than SOP,
such as the third-party mashup, cannot be implemented in JSFlow.
%
%%Second, developers need to understand a new semantics for the core
%%JavaScript language~\cite{Hedin:2012}, in contrast to a new DOM-leve
%%API.
%
Second, library APIs, including the JavaScript built-in APIs and the
DOM, must be individually modeled.
%
This not only restricts the APIs available to developers, but also
incurs a huge performance cost (by two orders of magnitude).
%
While the goal of JSFlow has been to understand the foundation for
fine-grained IFC for JavaScript, as opposed to providing a deployable
solution, we believe that their work is complimentary.
%
Indeed, one can envision a system where some \sys{} compartments
enforce fine-grained IFC by leveraging JSFlow (which may itself use
the \sys{} API).


%% Hedin and Sabelfeld~\cite{Hedin:2012} formally describe a sound language-based
%% confinement mechanisms for a subset of JavaScript and their ideas are being
%% currently applied to JSFlow~\cite{JSFlow}. %, a modified JavaScript interpreter.
%% JSFlow is designed to confine legacy code, and because of that, it does an
%% impressive effort to obtain fine-grain labeling of data. This is achieved at the
%% price of proposing detailed models for capturing big parts of the browser
%% semantics (e.g. interaction with the DOM), which drastically impact on
%% performance (sometimes more than 100\%!).  We provide a more coarse-grain
%% approach: \emph{\sys{} does not care for most of the JavaScript and DOM API
%%   internals}. Instead, \sys~ uses the browsing contexts and simply
%% mediates among them to preserve security.



% With our
FlowFox~\cite{DeGroef:2012} is an IFC system atop Firefox, based on
secure-multi execution (SME)~\cite{Devriese:2010}.
%
FlowFox executes a program multiple times, once per each label; this
SME approach ensures that no leaks from a sensitive context can leak
into a less-senstive context by construction.
%
Similar to JFlow, FlowFox can confine programs wherein data is labeled
at a fine-grained level.
%
In turn, FlowFox associates labels with user interaction and meta data
(history, cookies, screen size, etc.) to address common privacy
attacks such as history sniffing behavior tracking; these attacks are
outside the scope of \sys{}.
%
Converesely, FlowFox's SME appraoch is not amendable to scenarios of
mutual distrust where declassification plays a key role; hence, they
do not consider new applications such as the encrypted editor or
third-party masup.

%
%% Authors show how FlowFox can enforce
%% non-interference like policies for popular web cites. While \sys{} does not
%% target legacy code, it can enforce a wider-range of policies
%% (e.g. declassification). Different from \sys{}, FlowFox requires a total
%% ordering of the security lattice, an uncommon assumption in an scenario with
%% mutual distrust (as the web). While recent results show how to lift this
%% requirement in reactive systems (as the browser)~\cite{ZanariniJR13}, FlowFox
%% has not yet incorporate them into its design.

% ConDOM 
\todo{ar}{revise as the above}
ConDOM~\cite{ConDOM} implements fine-grained label tracking system
that spans both the JavaScript engine and DOM.
%
To handle implicit flows, authors use a
control flow stack (neglecting exceptions), and inject labels at the HTML parser
for dynamically generated code. Instead, \sys{} allows to handle any type of
branches or dynamic code by just using browsing contexts.


%Sandboxing/language subsets
There is much work on sandboxing and developing subsets of JavaScript (e.g.,
Caja~\cite{GoogleCaja}, BrowserShield~\cite{Reis:2007},
WebJail~\cite{VanAcker:2011}, TreeHouse~\cite{Ingram:2012},
JSand~\cite{Agten:2012:JCC}, SafeScript~\cite{SafeScript}, Defensive
JavaScript~\cite{djs}). 
%
While some of these systems (e.g., TreeHouse) have inspired our design, the
underlying high-level ideas behind these systems are to mediate security
critical operations, restrict access to the DOM, and restrict communication APIS.
%
In contrast to the mandatory nature of confinement, however, most restrictions
are imposed in a discretionary fashion and are thus not suitable to the
building a number of the applications we consider (in particular, the encrypted
editor).
%
Nevertheless, we believe that access control and language subsets are a crucial
complement to confinement when building robust secure applications.



% Google extensions security analysis 
Carlini \emph{et al.}~\cite{Carlini:2012} evaluate the security of the
Google Chrome extension platform.
%
\sys{}'s lightweight DOM workers are very similar to content scripts,
though provided as first-class DOM object.
%
Different from real extensions, \sys{} does not not consider
privileged APIs, and we do not expect to extend our API, in the near
further.
%
Instead, we believe that extension systems can stand to benefit from
providing confinement to content scripts, if only to provide a means
for further reducing the trust placed on existing extensions.
%
Indeed, since Firefox content scripts rely on the same core mechanisms
used by \sys{} we expect a port of AddOn-SDK to use confinement to be
achievable with modest effort.
 

% Local Variables:
% TeX-master: "main.ltx"
% TeX-command-master: "make"
% tex-dvi-view-command: "gmake preview;:"
% End:
