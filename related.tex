\section{Related Work}
\label{sec:related}

% BFlow
\Red{BFlow}
A closely related work is BFlow~\cite{Yip:2009:PBS}.
%
Like \sys{}, BFlow allows websites to enforce confinement policies
that are stricter than the SOP.
%
BFlow can, for example, be used to confine third-party code such as
the password strength checker by placing the untrusted code in a
tainted \emph{protection zone}, which is a group of iframes that share
a common label.
%
However, unlike \sys{}, BFlow is not amendable to a mutually
distrusting scenario and thus new applications such as the encrypted
document editor and password manager cannot be implemented without
giving up trust.
%
This stems from two design design points.
%
First, BFlow uses hierarchical confinement: the label of a subframe
(of a different protection zone) must have a more restricting label
than its parent; hence, an frame cannot impose any restrictions on its
parent.
%
Second, BFlow labels cannot encode the fact that data may be sensitive
to two websites and thus application, such as the encrypted editor and
password manager, that require one party to partially declassify a
piece of information cannot be expressed.
%
(For declassification, BFlow relies on a top-level \emph{trusted zone}
that can always \emph{completely} declassify the data in the page.)
%
This latter design point further prevents BFlow from being amendable
to building applications, such as our third-party mashup, that rely on
a policy more flexible than the SOP.
%
In fairness, BFlow did not consider applications with mutually
distrusting parties---their goal was to confine untrusted third-party
scripts, and indeed succeed at doing so.
%
Moreover, while \sys{} propagates labels using HTTP headers and is
amendable to confining one user's code from another on the same
website, we do not provide a server that implements this
functionality; developers can use BFlow's server-side component or
systems such as Hails~\cite{giffin:2012:hails} and
Aeolus~\cite{cheng:aeolus} to ensure that labels are propagated
appropriately in an end-to-end fashion.
%
\Red{I think the story might be even better for us: I don't think they
can build password checker where the latter is an iframe hosted on
other domain.}


% DCS 
\Red{DCS}
More recently, Akhawe \emph{et al.} proposed extending the browser
with data-confined sandboxes (DCS)~\cite{Akhawe2013}.
%
DCS provides pages with the ability to intercept and monitor the
network, storage and cross-origin channels of iframes that have
\verb|data:| URI.
%
This, in turn, can be used used to confine code such as the password
strength checker.
%
Unfortunately, limiting the confinement to \verb|data:| URI iframes
means that DCS cannot be used to confine a password strength checker
service, i.e., checker supplied in the form of an iframe by a
third-party.
%
Like BFlow, DCS address the third-party library confinement problem
and thus does not have the symmetry property of \sys{}---an iframe
cannot impose restrictions on the parent---that is crucial for
building new applications such as the encrypted editor or password
manager.
%
Unlike \sys{}, DCS is not limited to using labels for policy
specification; instead, it allows a page to implement policies using
arbitrary JavaScript.
%
However, to ensure that the monitor cannot be used to learn more
information about the iframe content than allowed by the SOP, the
kinds of policies that a page can impose on an iframe is limited
(e.g., DCS cannot impose restrictions on redirects).
%
\sys{} does not have this limitation since the iframe itself imposes
the policy (albeit as a result of wishing to receive sensitive data
from the parent).
%
This design choice of using declarative policies further allows us to
relax the SOP and build applications such as the third-party mashup.
%
However, while we believe that using labels is sufficient for many
web applications, we recognize this as a limitations of \sys{} and
leave the investigation of expressiveness to future work.
\Red{It's not clear if an iframe can monitor a sub-iframe}

%% Recently, and different from other sandboxes approaches, the
%% data-confined sandbox system~\cite{Akhawe2013} (DCS) restricts
%% propagation of information by using \js|iframes| and mediating
%% cross-domain operations (e.g. access to local storage, fragment-IDs,
%% network communication, etc.).  Every sandbox communicates by
%% \js!postMessage! only with its designated parent. The parent defines
%% the confinement policy and plays the role of man-in-the-middle when a
%% sandbox wish to communicate with another one. While having similar
%% goals, DCS is more restrictive than \sys. DCS disallows dynamic code
%% evaluation inside sandboxes and presents a not egalitarian design,
%% i.e., a sandbox (\js|iframe|) cannot host another sandbox. Clearly,
%% the DCS client-side TCB is smaller than ours.

%% BFlow~\cite{Yip:2009:PBS} is a confinement system for web browsers. It tracks
%% flows of information at the granularity of \emph{secure zones}, i.e.,
%% compartments composed of one or several iframes.  {\sys} does not require such
%% abstraction, it relies on the compartment notions already provided by browsers,
%% i.e., iframes and workers. Similar to our approach, BFlow uses \js|postMessage|
%% for communication across secure zones. BFlow does not allow JavaScript code in
%% different secure zones to write to shared DOM variables and cookies regardless
%% their security labels.  While \sys~ supports more expressive labels, potentially
%% including different domains, BFlow can propagates labeled data to the server
%% side. It is stated as future work to achieve a similar feature.



% Javascript Sandboxing: JSand and TreeHouse
%Although less related, 
Language-based approaches for sandboxing 
JavaScript code have been a topic of recent research interest
(e.g. Caja~\cite{GoogleCaja}, BrowserShield~\cite{Reis:2007}, WebJail~\cite{VanAcker:2011}, 
TreeHouse~\cite{Ingram:2012}, JSand~\cite{Agten:2012:JCC}, SafeScript~\cite{SafeScript}, etc.).
%Focusing on different security policies, several language-based approaches has
%been divised to sandbox Javascript.  
These proposals usually work in a similar manner: they mediate security critical
operations as well as some of the browser's API (e.g. the DOM). We note that
sandboxing policies are naturally different from confinement. \sys{}'s notion of
cleareance, however, provides a notion of sandboxing, i.e., browsing
contexts about a given clearance cannot be accessed.

 
% % JSand
% JSand~\cite{Agten:2012:JCC} wraps setters (\js|set|) and getters (\js|get|) of Javascript
% objects. Additionally, it propagates policies to newly created objects via a
% membrane pattern~\cite{RobustComposition}. Application-specific DOM-nodes are freely accessed
% withtin the sandbox. For global properties (e.g. \js|window.document|), JSand
% simply wraps the \js|window| object in order to enforce a given policy.
% % TreeHouse
% Similar to our work, TreeHouse~\cite{Ingram:2012} uses workers to provide a
% fresh and isolated execution context for Javascript code. Workers communicate by
% \js|postMessage|, which guarantees that no references to outside objects can be
% passed into the sandbox. TreeHouse provides virtualized DOM-nodes with the
% restriction that workers cannot share them.
% % Limitations
% Aiming to target legacy code, JSand and TreeHouse present some limitations
% related to performance and completeness.
% % Performance
% While exposing significant performance degradation in micro benchmarking (in
% some cases more than 400\%!), authors claim that the impact on the user
% experience is acceptable.
% % Completeness
% On the completeness side, JSand parses dynamically loaded scripts either
% asynchronously, potentially changing the semantics of the application, or
% through a library, which might interpret them differently from the
% browser. Similarly, TreeHouse changes a synchronous call semantics to the DOM
% API by an asynchronous one.


% JSFlow 
%Tracking IFC at the granularity level of JavaScript instructions is indeed
%challenging. The main reason for that being the ramification of side-effects
%triggered when evaluation scripts (e.g., implicit coercions, live collections
%updates, and others). 
Hedin and Sabelfeld~\cite{Hedin:2012} formally describe a sound language-based
confinement mechanisms for a subset of JavaScript and their ideas are being
currently applied to JSFlow~\cite{JSFlow}. %, a modified JavaScript interpreter.
JSFlow is designed to confine legacy code, and because of that, it does an
impressive effort to obtain fine-grain labeling of data. This is achieved at the
price of proposing detailed models for capturing big parts of the browser
semantics (e.g. interaction with the DOM), which drastically impact on
performance (sometimes more than 100\%!).  We provide a more coarse-grain
approach: \emph{\sys{} does not care for most of the JavaScript and DOM API
  internals}. Instead, \sys~ uses the browsing contexts and simply
mediates among them to preserve security.
% With our
% approach, webpage developers can obtain fine-grain IFC by restructuring their
% code for security reasons.

% ConDOM 
Using WebKit, ConDOM~\cite{ConDOM} implements fine-grain label tracking at the
JavaScript engine and DOM elements. To handle implicit flows, authors use a
control flow stack (neglecting exceptions), and inject labels at the HTML parser
for dynamically generated code. Instead, \sys{} allows to handle any type of
branches or dynamic code by just using browsing contexts.

% Secure multi-execution in the browser? 
FlowFox~\cite{DeGroef:2012} adapts secure-multi execution~\cite{Devriese:2010}
(SME) for the web scenario. Authors show how FlowFox can enforce
non-interference like policies for popular web cites. While \sys{} does not
target legacy code, it can enforce a wider-range of policies
(e.g. declassification). Different from \sys{}, FlowFox requires a total
ordering of the security lattice, an uncommon assumption in an scenario with
mutual distrust (as the web). While recent results show how to lift this
requirement in reactive systems (as the browser)~\cite{ZanariniJR13}, FlowFox
has not yet incorporate them into its design.




% Google extensions security analysis 
Carlini et al.~\cite{Carlini:2012} evaluate the security of the Google Chrome
extension platform. The authors found several vulnerabilities in extensions, among them,
how content scripts can inject arbitrary code (in a form of strings) into the
page DOM. Extensions can then run within the page context in order
to exfiltrate data. \sys{} is capable to securely mimicking content scripts, but
with  weaker isolation guarantees.  Specifically, content scripts operate on a
copy of DOM wrapper objects, which makes changes not visible for the
page. Instead, \sys{} versions of content scripts work directly on the page DOM.
It is stated as future work to evaluate how to express other 
security mechanisms from the Google Chrome extension platform into \sys{}. 

\todo{pm/bk}{scriptpolice}
 
 

% Local Variables:
% TeX-master: "main.ltx"
% TeX-command-master: "make"
% tex-dvi-view-command: "gmake preview;:"
% End:
