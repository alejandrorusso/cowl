\section{Related Work}
\label{sec:related}



% Javascript Sandboxing: JSand and TreeHouse
Focusing on different security policies, several language-based approaches has
been divised to sandbox Javascript.  These approaches usually work in a similar
manner: they mediate security critical operations and some of the browser API
(e.g. the DOM).
% JSand
JSand~\cite{Agten:2012:JCC} wraps setters (\js|set|) and getters (\js|get|) of Javascript
objects. Additionally, it propagates policies to newly created objects via a
membrane pattern~\cite{RobustComposition}. Application-specific DOM-nodes are freely accessed
withtin the sandbox. For global properties (e.g. \js|window.document|), JSand
simply wraps the \js|window| object in order to enforce a given policy.
% TreeHouse
Similar to our work, TreeHouse~\cite{Ingram:2012} uses workers to provide a
fresh and isolated execution context for Javascript code. Workers communicate by
\js|postMessage|, which guarantees that no references to outside objects can be
passed into the sandbox. TreeHouse provides virtualized DOM-nodes with the
restriction that workers cannot share them.
% Limitations
Aiming to target legacy code, JSand and TreeHouse present some limitations
related to performance and completeness.
% Performance
While exposing significant performance degradation in micro benchmarking (in
some cases more than 400\%!), authors claim that the impact on the user
experience is acceptable.
% Completeness
On the completeness side, JSand parses dynamically loaded scripts either
asynchronously, potentially changing the semantics of the application, or
through a library, which might interpret them differently from the
browser. Similarly, TreeHouse changes a synchronous call semantics to the DOM
API by an asynchronous one.


%


% Local Variables:
% TeX-master: "main.ltx"
% TeX-command-master: "make"
% tex-dvi-view-command: "gmake preview;:"
% End:
