\section{Related Work}
\label{sec:related}

Existing implementations of information flow control in the browser
can be classified either as \emph{fine-grained} or
\emph{coarse-grained}. The former associate IFC policies with
individual objects, while the latter associate policies with entire
browsing contexts. We compare \sys{} to previously proposed systems
in both categories, then contrast the two categories' overall
characteristics.
%% We first first compare \sys{} to other
%% coarse-grained systems, and then discuss how it compares with
%% fine-grained systems.

\paragraph{Coarse-grained IFC} \sys{} shares many features
with existing coarse-grained systems.
% BFlow
%
BFlow~\cite{Yip:2009:PBS}, for example, allows web sites to enforce confinement policies
stricter than the SOP via \emph{protection zones}---groups of iframes sharing a common label.
%
%   BFlow can confine third-party code by placing
%   it in a tainted \emph{protection zone}---a group
%   of iframes that share a common label.
%
However, BFlow cannot mediate between mutually
distrustful principals---\emph{e.g.,} the
encrypted document editor is not directly implementable with BFlow.
%
This is because only asymmetric confinement is supported---a
sub-frame cannot impose any restrictions on its parent.
%
For the same reasons, BFlow cannot support applications that require
security policies more flexible than the SOP, such as our third-party
mashup example.
%
These differences reflect different goals for the two systems. BFlow's
authors set out to confine untrusted third-party scripts, while we
also seek to support applications that incorporate code from mutually
distrusting parties.


% DCS 
More recently, Akhawe \emph{et al.}~propose the data-confined sandbox
(DCS) system~\cite{Akhawe2013}, which allows pages to intercept and
monitor the network, storage, and cross-origin channels of \verb|data:| URI iframes.
%
The limitation to
\verb|data:| URI iframes means DCS cannot confine the
common case of a service provided in an iframe~\cite{postman}.
%% ---the common case, both to isolate the
%% enclosing page from the script and because the checker's author may
%% not want to share source code
%
Like BFlow, DCS does not offer symmetric confinement, and does
not incorporate functionality to let developers build
applications like third-party mashups.
%% ---an iframe
%% cannot impose restrictions on its parent. It therefore cannot support
%% applications that combine code from mutually distrusting origins, as
%% does the encrypted document editor example.
%
%% DCS further does , which require safe relaxation of the
%% SOP\@.

\paragraph{Fine-grained IFC}
%% Fine-grained IFC systems differ from \sys{} in that they track
%% information flow at the level of objects, allowing developers to take
%% advantage of information flow control without explicit
%% compartmentalization of applications.
Per-object-granularity IFC makes
it easier to confine untrusted libraries that are closely coupled with trusted
code on a page (\emph{e.g.,} jQuery) and avoid the problem of
\emph{over-tainting,}
%or \emph{label creep},
where a single context accumulates taint as it inspects more data.

% JSFlow 
JSFlow~\cite{JSFlow} is one such fine-grained JavaScript IFC system, which
enforces policies by executing JavaScript in an interpreter written in
JavaScript.
%
This approach incurs a two order of magnitude slowdown. JSFlow's
authors suggest that this cost makes JSFlow a better fit for use as a
development tool than as an ``always-on'' privacy system for users'
browsers.
%
Additionally, JSFlow does not support applications that rely on policies
more flexible than the SOP, such as our third-party mashup example.

% With our
The FlowFox fine-grained IFC system~\cite{DeGroef:2012} enforces
policies with secure-multi execution (SME)~\cite{Devriese:2010}. SME
ensures that no leaks from a sensitive context can leak into a less
sensitive context by executing a program multiple times.
%
Unlike JSFlow and \sys{}, SME is not amenable to
scenarios where declassification plays a key role (\emph{e.g.,} the encrypted
editor or the password manager).
%
FlowFox's labeling of user interactions and metadata (history, screen
size, \emph{etc.}) do allow it to mitigate history sniffing and
behavior tracking; \sys{} does not address these attacks.
%% , where
%% developers must impose policy on the data to be shared.

While fine-grained IFC systems may be more convenient for developers,
they impose new language semantics for developers to learn, require
invasive modifications to the JavaScript engine, and incur greater
performance overhead.
%
In contrast, because \sys{} repurposes familiar isolation constructs
and does not require JavaScript engine modifications, it is relatively
straightforward to add to legacy browsers. It also only adds overhead
to cross-compartment operations, rather than to all JavaScript
execution. The typically short lifetime of a browsing context helps
avoid excessive accumulation of taint.
%
%% We believe that coarse-grained IFC suffers few limitations in
%% practice.
%While there is a difference in programmer convenience,
We conjecture that coarse-grained and fine-grained IFC are equally
expressive, provided one may use arbitrarily many compartments---a
cost in programmer convenience.
%
Finally, coarse- and fine-grained mechanisms are not mutually
exclusive. For instance, to confine legacy (non-compartmentalized)
JavaScript code, one could deploy JSFlow within a COWL context.
%% to confine legacy (non-compartmentalized) JavaScript code in a more
%% fine-grained fashion.

\paragraph{Sandboxing}
The literature on sandboxing and secure subsets of JavaScript is rich,
and includes Caja~\cite{GoogleCaja}, BrowserShield~\cite{Reis:2007},
WebJail~\cite{VanAcker:2011}, TreeHouse~\cite{Ingram:2012},
JSand~\cite{Agten:2012:JCC}, SafeScript~\cite{SafeScript}, Defensive
JavaScript~\cite{djs}, and Embassies~\cite{Howell:2013}).
%
While our design has been inspired by some of these systems (\emph{e.g.,}
TreeHouse), the usual goals of these systems are to mediate
security-critical operations, restrict access to the DOM, and restrict
communication APIs\@.
%
In contrast to the mandatory nature of confinement, however, these
systems impose most restrictions in discretionary fashion, and are
thus not suitable for building some of the applications we consider
(in particular, the encrypted editor).
%
Nevertheless, we believe that access control and language subsets are
crucial complements to confinement for building robustly secure
applications.

% Local Variables:
% TeX-master: "main.ltx"
% TeX-command-master: "make"
% tex-dvi-view-command: "gmake preview;:"
% End:
