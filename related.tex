\section{Related Work}
\label{sec:related}

% Javascript Sandboxing: JSand and TreeHouse
%Although less related, 
Language-based approaches for sandboxing 
JavaScript code have been a topic of recent research interest
(e.g. Caja~\cite{GoogleCaja}, BrowserShield~\cite{Reis:2007}, WebJail~\cite{VanAcker:2011}, 
TreeHouse~\cite{Ingram:2012}, JSand~\cite{Agten:2012:JCC}, SafeScript~\cite{SafeScript}, etc.).
%Focusing on different security policies, several language-based approaches has
%been divised to sandbox Javascript.  
These proposals usually work in a similar manner: they mediate security critical
operations as well as some of the browser's API (e.g. the DOM). We note that
sandboxing policies are naturally different from confinement. \sys{}'s notion of
cleareance, however, provides a notion of sandboxing, i.e., browsing
contexts about a given clearance cannot be accessed.

 
% % JSand
% JSand~\cite{Agten:2012:JCC} wraps setters (\js|set|) and getters (\js|get|) of Javascript
% objects. Additionally, it propagates policies to newly created objects via a
% membrane pattern~\cite{RobustComposition}. Application-specific DOM-nodes are freely accessed
% withtin the sandbox. For global properties (e.g. \js|window.document|), JSand
% simply wraps the \js|window| object in order to enforce a given policy.
% % TreeHouse
% Similar to our work, TreeHouse~\cite{Ingram:2012} uses workers to provide a
% fresh and isolated execution context for Javascript code. Workers communicate by
% \js|postMessage|, which guarantees that no references to outside objects can be
% passed into the sandbox. TreeHouse provides virtualized DOM-nodes with the
% restriction that workers cannot share them.
% % Limitations
% Aiming to target legacy code, JSand and TreeHouse present some limitations
% related to performance and completeness.
% % Performance
% While exposing significant performance degradation in micro benchmarking (in
% some cases more than 400\%!), authors claim that the impact on the user
% experience is acceptable.
% % Completeness
% On the completeness side, JSand parses dynamically loaded scripts either
% asynchronously, potentially changing the semantics of the application, or
% through a library, which might interpret them differently from the
% browser. Similarly, TreeHouse changes a synchronous call semantics to the DOM
% API by an asynchronous one.


% JSFlow 
%Tracking IFC at the granularity level of JavaScript instructions is indeed
%challenging. The main reason for that being the ramification of side-effects
%triggered when evaluation scripts (e.g., implicit coercions, live collections
%updates, and others). 
Hedin and Sabelfeld~\cite{Hedin:2012} formally describe a sound language-based
confinement mechanisms for a subset of JavaScript and their ideas are being
currently applied to JSFlow~\cite{JSFlow}, a IFC-aware JavaScript interpreter.
JSFlow is designed to confine legacy code, and because of that, it does an
impressive effort to obtain fine-grain labeling of data. This is achieved at the
price of proposing detailed models for capturing big parts of the browser
semantics (e.g. interaction with the DOM), which drastically impact on
performance (sometimes more than 100\%!).  We provide a more coarse-grain
approach: \emph{\sys{} does not care for most of the JavaScript and DOM API
  internals}. Instead, \sys~ uses IFC-aware browsing contexts and simply
mediates among them to preserve security.
% With our
% approach, webpage developers can obtain fine-grain IFC by restructuring their
% code for security reasons.

% ConDOM 
Using WebKit, ConDOM~\cite{ConDOM} implements fine-grain label tracking at the
JavaScript engine and DOM elements. To handle implicit flows, authors use a
control flow stack (neglecting exceptions), and inject labels at the HTML parser
for dynamically generated code. Instead, \sys{} allows to handle any type of
branches or dynamic code by just using browsing contexts.

% BFlow
BFlow~\cite{Yip:2009:PBS} is a confinement system for web browsers. It tracks
flows of information at the granularity of \emph{secure zones}, i.e.,
compartments composed of one or several iframes.  {\sys} does not require such
abstraction, it relies on the compartment notions already provided by browsers,
i.e., iframes and workers. Similar to our approach, BFlow uses \js|postMessage|
for communication across secure zones. BFlow does not allow JavaScript code in
different secure zones to write to shared DOM variables and cookies regardless
their security labels.  While \sys~ supports more expressive labels, potentially
including different domains, BFlow can propagates labeled data to the server
side. It is stated as future work to achieve a similar feature.

% Secure multi-execution in the browser? 
FlowFox~\cite{DeGroef:2012} adapts secure-multi execution~\cite{Devriese:2010}
(SME) for the web scenario. Authors show how FlowFox can enforce
non-interference like policies for popular web cites. While \sys{} does not
target legacy code, it can enforce a wider-range of policies
(e.g. declassification). Different from \sys{}, FlowFox requires a total
ordering of the security lattice, an uncommon assumption in an scenario with
mutual distrust (as the web). While recent results show how to lift this
requirement in reactive systems (as the browser)~\cite{ZanariniJR13}, FlowFox
has not yet incorporate them into its design.




% DCS 
Recently, and different from other sandboxes approaches, the data-confined
sandbox system~\cite{Akhawe2013} (DCS) restricts propagation of information by using
\js|iframes| and mediating cross-domain operations (e.g. access to local
storage, fragment-IDs, network communication, etc.).  Every sandbox communicates
by \js!postMessage! only with its designated parent. The parent defines the
confinement policy and plays the role of man-in-the-middle when a sandbox wish
to communicate with another one. While having similar goals, DCS is more
restrictive than \sys. DCS disallows dynamic code evaluation inside sandboxes
and presents a not egalitarian design, i.e., a sandbox (\js|iframe|) cannot host
another sandbox. Clearly, the DCS client-side TCB is smaller than ours.


% Google extensions security analysis 
Carlini et al.~\cite{Carlini:2012} evaluate the security of the Google Chrome
extension platform. The authors found several vulnerabilities in extensions, among them,
how content scripts can inject arbitrary code (in a form of strings) into the
page DOM. Extensions can then run within the page context in order
to exfiltrate data. \sys{} is capable to securely mimicking content scripts, but
with  weaker isolation guarantees.  Specifically, content scripts operate on a
copy of DOM wrapper objects, which makes changes not visible for the
page. Instead, \sys{} versions of content scripts work directly on the page DOM.
It is stated as future work to evaluate how to express other 
security mechanisms from the Google Chrome extension platform into \sys{}. 


 
 

% Local Variables:
% TeX-master: "main.ltx"
% TeX-command-master: "make"
% tex-dvi-view-command: "gmake preview;:"
% End:
