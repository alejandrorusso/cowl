\section{The \sys{} confinement system}
\label{sec:system}

In this section, we describe the \sys{} system.
%
Throughout, we'll refer to our running example, the \emph{password
checker}, which demonstrates many of the key design points of \sys{}.

\subsection{Compartments and origins}

%   Browsing contexts. As shown in Figure 1, a web page
%   consists of content and JavaScript code. A browsing con-
%   text presents a page’s content to the user, and JavaScript
%   code accesses content within the page through the Docu-
%   ment Object Model (DOM) [34]. Browsing contexts may
%   be nested (e.g., by using iframes). 

% rectangular frames denote
% browser contexts

% Contexts may be labeled with the ori-
%gins to whose sensitive data they have been exposed. A
%context that has not yet observed sensitive data is de-
%noted public , and a page that contains sensitive data
%can raise the label of its own browsing context to its own
%origin.

As mentioned previously, the basic unit of confinement in \sys{}---a
compartment---corresponds to a JavaScript execution context, e.g.\@ a
browser tab, an iframe on a page, or even a Web Worker~\cite{workers}.
%
Our password checker was partitioned into two compartments: the main
page and an untrusted worker which ran the actual password checker code
(though an iframe could easily have been used in place of the worker).
%
Each compartment and web site is associated with a security policy which
dictates what kind of messages and requests are permitted to flow with
other compartments and web sites.
%They also may read
%   and write persistent storage (e.g., cookies) and issue net-
%   work requests (either implicitly in page content that ref-
%   erences a URL retrieved over the network, or explicitly
%   in JavaScript, through invoking the XMLHttpRequest
%   (XHR) constructor).
\sys{}'s job is to enforce this policy on all forms of communication, from
{\tt postMessage}s to mutation of a shared DOM\@.\footnote{While shared DOMs
usually occur in the context of iframes, we also introduce a new type
of Web Worker called a DOM Worker which shares a DOM with its parent.
These workers run synchronously with the parent (rather than in parallel)
and are solely used for confinement: one can think of them as iframes
without DOMs of their own.}
%
Compartments can create new compartments: \sys{} extends the relevant
APIs to also accept the security policy which should be applied to the
new compartment (with some restrictions).
%
We describe the structure of our security policies in the next section.

\subsection{Labels}
\label{sec:labels}

A \emph{label} is used to encode security policies in our system.
%
Conceptually, a label protects all data associated with a compartment or
web site, specifying what origins can read the data.
%
Individual pieces of data can also be labeled, see Section~\Red{labeled}.
%
The particular labels that are used in \sys{} are \emph{DC labels},
presented in~\cite{stefan:2011:dclabels}.
%
We chose DC labels, much like Hails~\cite{giffin:2012:hails} and
Breeze~\cite{Breeze13}, because they have a simple and clear
semantics, yet are as expressive as the labels used by other practical
confinement systems~\cite{GenLabels}.
%
In this section, we give a brief description of DC labels; an interesting
reader should see~\cite{stefan:2011:dclabels} for more details.

A privacy label (henceforth just label) is a boolean formula over
origins.%
%
\footnote{
  Our system also handles \emph{integrity}, but it is omitted from
  the discussion for simplicity.
}
%
When a compartment wishes to send a (unidirectional) message to another compartment or
web site, we check if the target label $l'$ logically implies the source
label $l$ ($l' \rightarrow l$): intuitively, this require that the security of the destination be \emph{as strong as} original security policy.\footnote{If the communication is bidirectional, e.g.\ mutation of a shared DOM, then both the target and source must imply each other.}
%
Thus, when the implication holds, we also say that the target label
\emph{subsumes} the source label.
%
As a simple example, in the password checker, the labeled password
cannot be read by a publically labeled worker (corresponding to the
logical formula ``true''), because ``true'' does not imply
``\https{fb.us}''.
%
However, once the worker raises its label to ``\https{fb.us}'', the
logical implication holds trivially.

%Reworded version: Intuitively, an entity is allowed to receive data that is at most as
%sensitive as its label, i.e., the entity's label always \js|subsumes|
%the label of the data it can receive, preserving the latter's privacy.
%   \Red{maybe something about how this makes it first
%   class: the fact that policies are preserved means that if a library has
%   sub-libraries, they will get the appropriate policy enforced}
%
Why is a label represented as a boolean formula rather than simply
as a set of origins?
%
As it turns out, both conjunctions (``and'')
and disjunctions (``or'') are useful for specifying security policies.
%
Informally, a conjunction states what origins contributed sensitive data
to the compartment, while a disjunction states what origins are allowed
to read the data.
%
% EZY: Unfortunately, I can't use the password checker here because
% the labels are too simple
For example, a worker that processes your bank statement and your shopping
history might have the label ``\https{bank.ch} and \https{amazon.com}''.
%
While it can read messages from either origin, it cannot send data to
\https{bank.ch}, as doing so might leak private information from
\https{amazon.com} (and vice versa).\footnote{Note that even initiating a
GET request constitutes sending data to an origin.}
%
Conversely, a compartment labeled ``\https{bank.ch} or \https{amazon.com}''
could send data to either website; however, it could not receive any
messages from either origin.

When a compartment is created, it is either explicitly given a label
(which must subsume the parent's label) or implicitly inherits the label
of its parent.\footnote{Perceptive readers may now be wondering how
the password checker arranged to create a publically labeled worker
when its label was \https{fb.us}.  An explanation will have to wait
until Section~\ref{sec:privileges}, where we talk about privileges.}
%
If there was no parent, then this compartment was a web page served
by the web site, in which case we look for a {\tt Document-label} header
specifying the initial label or default to the origin of the web site.
%
A compartment can also freely \emph{raise} its label, i.e.\ change its label
to any label which subsumes it, but not vice versa.
%
This leads to a common interpretation of labels as ``taint;'' as
a compartment receives data from more sources, it accumulates more
conjuncts on account of raising its label to read the received data.

\subsection{Privileges}
\label{sec:privileges}

% Description of privileges
A compartment is associated with a label representing its \emph{privilege}.
%
Privileges confer the authority to relax policy requirements
with respect to the origins in the label.
%
This is useful when you have some trusted code which should be
able to lower the current label or send a message with a lower label, so
the recipient does not need to raise its label in order to read it.
%
Privileges are exercised implicitly: a compartment with a privilege $p$
adjusts a label checks from $l$ to $l'$ by requiring that $l' \land p
\rightarrow l$.

% Basic usage patterns of privileges
Privileges offer another dimension of separation for compartments: two
compartments might have the same label, but a trusted compartment might
have a privilege while an untrusted compartment does not.
%
The trusted compartment then has the ability to declassify data (making
it public or sending it to a cross-origin website) or create data
that is disjunctively labeled.
%
That is, with the privilege \https{amazon.com}, I can create data with
the label ``\https{amazon.com} or \https{bank.ch}'': this
data can be sent to either web site but not to any other site.

% Initial privileges and privilege creation
By default, a browsing context has the privilege corresponding to the
page origin; this context can drop the privilege if it is not
needed or spawn other containers sans privilege to run untrusted code.
%
While privileges for existing origins cannot be created (that would be
forgery), a privilege can be created for a fresh origin, i.e.\ an origin
which does not correspond any actual web site, but which is guaranteed
to be unique.
%
A fresh origin is often used to run code under full isolation, since any
compartment labeled with the fresh origin cannot communicate with any
other compartment without exercising the corresponding privilege.
%
Importantly, any code can allocate a origin-privilege pair: this
makes isolation an egalitarian security mechanism: an untrusted library
can use this mechanism to isolate a sub-library that it itself
may not trust~\cite{Zeldovich:2006}.

\subsection{Labeled blobs}
\label{sec:labeled-blobs}

A \emph{labeled blob} is an arbitrary piece of data with a label
associated with it.
%
A programmer can use labeled blobs to express security policy at
a more fine-grained level than per-compartment.
%
Labeled blobs can be passed around as an opaque object (in which case one merely
checks the security policy of the compartments involved).
%
However, when a piece of code wishes to inspect the contents of the
labeled blob, a label check between the label of the blob and the
current compartment occurs.

Labeled blobs can be explicitly created by the programmer, and then
passed around by container-to-container message passing.
%
However, a compartment can also make a request to a website using
\emph{labeled XHR} and received a labeled blob, without raising their
label read the data.
%
This effectively allows a developer to defer reading the result of
a request: this is useful if the compartment merely plans on passing
the labeled blob to another compartment, or if it needs to query
multiple websites without raising its label (otherwise, upon reading
its first response, it would raise its label and be unable to
make any further requests.)

\Red{I dunno, maybe some comments about how this pattern shows up
in some web APIs, e.g.\ service workers}

% maybe footnote about integrity

\subsection{Clearance}

In an ideal confinement system, code can read arbitrary data at the
cost of raising the context label (and thus giving up write
capabilities).
%
Unfortunately, practical systems typically have covert channels which
may be exploited to leak sensitive data.
%
Hence, as in HiStar~\cite{Zeldovich:2006}, Hails~\cite{giffin:2012:hails}, and
Breeze~\cite{Breeze13} we associate a \emph{clearance}---which is
simply a label---with every compartment.
%
Clearance imposes a limit on the kind of data a compartment can have,
i.e., it is an upper-bound on compartment label, and thus the kind of
data a piece of code executing in the compartment can access.

\Red{Clearance choice by user? We need to know what the default
    clearance is.  The attack model is kind of funny: we want to argue
    that covert channels are obvious (e.g. crashing your browser) or low
    bit-rate; i.e. set in such a way that opting in to run the untrusted
code on the sensitive data is materially less harmful than if the
untrusted code ran automatically}

%   In \sys{}, the clearance assigned to browsing contexts is primarily a
%   choice made by the user.
%   %
%   ``Do I want to give this mash-up website (confined) access to my
%   bank statement?''  In absence of covert channels, the user's choice
%   makes no difference
%   \Red{Similarity to content scripts}

%   For completeness, we summarize  \sys{}'s API for compartments
%   in Figure~\ref{systemAPI}. In the next subsection, 
%   we explain the purpose of the different primitives through examples. 
%   We remark that \sys{}'s API is designed to be minimalistic, and as such, it only
%   consists on the primitives shown in Figures~\ref{fig:APIspec} and
%   \ref{systemAPI}.

%   In the rest of this section, we expand on the \sys{} design by
%   introducing mechanisms that meet the requirements of the four
%   example applications described in Section~\ref{sec:goals}.
%% through
%% three concrete examples: 
%% a password-strength checker (Section~\ref{sec:system:worker}),
%% a password manager (Section~\ref{sec:system:iframe}), 
%% a third-party mashup (Section~\ref{sec:system:mashup}), and
%% a library that converts phone numbers to links
%% (Section~\ref{sec:system:script}--Section~\ref{sec:system:extension}).
%
%For completeness, we summarize the different system components,
%security mechanisms, and DOM API (using WebIDL-like syntax), in
%Table~\toref{table:components}.



% Local Variables:
% TeX-master: "main.ltx"
% TeX-command-default: "Make"
% End:
