\section{The \sys{} Confinement System}
\label{sec:system}

The COWL confinement system extends the browser security model while
leaving the browser fully compatible with today's ``legacy'' web
applications.\footnote{In prior work, we described how confinement can
  subsume today's browser security primitives, and advocated replacing
  them entirely with a clean-slate, confinement-based
  model~\cite{yang:2013:towards}. In this paper, we instead prioritize
  incremental deployability, which requires coexistence alongside the
  status quo model.} Under COWL, the browser treats a page exactly
like a legacy browser does unless the page executes a COWL API
operation, at which point the browser records that page as running in
{\em confinement mode,} and all further operations by that page are
subject to confinement by COWL. COWL augments today's web browser with
three primitives, all of which appear in the simple password-checker
application example in Figure~\ref{fig:checker}.
% need to introduce "structural" primitives (ones that are visible in
% figures for example apps):
%
% labeled browsing contexts: in Figure 2. the central construct in
% COWL. they introduce MAC-based confinement to the browser. And they
% support hierarchical confinement, which is essential to allowing
% nesting of untrusted libraries (as described in section ...).

% privilege: in Figure 2 (though not used in example). in a MAC-based
% confinement system, certain operations are dangerous because
% untrusted code could use them to cause the release of sensitive
% information to unauthorized parties. for example, declassification
% of sensitive data strips the data's label, after which that data
% will not be confined within other compartments running potentially
% untrusted code that processes it. In COWL we term the authorization to execute
% such dangerous operations privilege, and express privilege in terms
% of origins.

% labeled communication: in Figure 2 (focus in that example chiefly on
% inter-context communication).  IPC between browsing contexts carries
% labeled data, so that when one compartment sends another sensitive
% data, it may confine the potentially untrusted code within the other
% compartment. This explicitly carrying of labels in IPC allows
% symmetric confinement, which is essential to building applications
% that compose scripts that are mutually distrusting, as described in
% section ....
%

{\em Labeled browsing contexts} enforce MAC-based confinement of
JavaScript at the granularity of a context (\emph{e.g.,} a worker or
iframe). The rectangular frames in Figure~\ref{fig:checker} are
labeled contexts. As contexts may be nested, labeled browsing
contexts allow hierarchical confinement, whose importance for
supporting nesting of untrusted libraries we discussed in
Section~\ref{sec:motivating-examples}.
%% Throughout this paper, when
%%describing a COWL-enabled browser, we use the terms compartment,
%%(browsing) context, and labeled (browsing) context interchangeably.
%% EZY is right---we say the above earlier!

When one browsing context sends sensitive information to another, a
sending context can use \emph{labeled communication} to confine the
potentially untrusted code receiving the information. This enables
symmetric confinement, whose importance in building applications that
compose mutually distrusting scripts we articulated in
Section~\ref{sec:motivating-examples}. In Figure~\ref{fig:checker},
the arrows between compartments indicate labeled communication, where
a subscript on the communicated data denotes the data's label.

COWL may grant a labeled browsing context one or more {\em
privileges,} each with respect to an origin, and each of which
reflects trust that the scripts executing within that context will not
violate the secrecy and integrity of that origin's data, \emph{e.g.,}
because the browser retrieved them from that origin. A privilege
authorizes scripts within a context to execute certain operations,
such as {\em declassification} and {\em delegation}, whose abuse would
permit the release of sensitive information to unauthorized parties.
In COWL, we express privilege in terms of origins. The crown icon in
the left compartment in Figure~\ref{fig:checker} denotes that this
compartment may execute privileged operations on data labeled with the
origin {\tt fb.com}---more succinctly, that the compartment holds the
privilege for {\tt fb.com}. The compartment uses that privilege to
remain unconfined by declassifying the checker response labeled {\tt
fb.com}.

%% A MAC system must restrict access to certain operations whose abuse
%% would permit the release of sensitive information to unauthorized
%% parties. We refer to such access-controlled operations in COWL as
%% privileged, and the right to invoke them as a {\em privilege.} For
%% example, if a script {\em declassifies} messages it receives from
%% another context, it will retain the ability to do as it pleases with
%% the data received in those messages---\emph{i.e.,} the sending context will
%% not have confined the receiving context. One must trust the script
%% invoking a privileged operation like declassification not to abuse
%% it. 

We now describe these three constructs in greater detail.

% need to state that MAC system and all primitives related to it are
% opt-in (and somewhere, how the developer opts in--what is the
% incantation?). Point is that we coexist alongside status quo
% browser, so we don't break any existing pages. It may be that
% incremental deployability is important enough to articulate as a
% goal in section 2. (we can potentially footnote that in our prior
% workshop paper, we explored how to implement all of today's browser
% security primitives in COWL-like primitives---a clean-slate
% approach that subsumed and improved on the status quo. here we take
% the complementary approach of prioritizing backward compatibility
% with existing mechanisms.)


\subsection{Labeled Browsing Contexts}
\label{sec:system:contexts}
A COWL application consists of multiple labeled contexts.
%
Labeled contexts extend today's browser contexts, used to isolate
iframes, pages, \emph{etc.}, with MAC \emph{labels}.
%
A context's label specifies the security policy for all data within
the context, which COWL enforces by restricting the flow of
information to and from other contexts and servers.
%% in addition to the restrictions imposed by existing DAC policies such
%% as the SOP.

As we have proposed previously~\cite{yang:2013:towards,
  stefan:2011:dclabels}, a label is a pair of boolean formulas over
  origins: a
\emph{secrecy} formula specifying which origins may read a context's
data, and an \emph{integrity} formula specifying which origins may
write it.
%
For example, only Amazon or Chase may read data labeled
\dcLabel{\https{amazon.com} $\lor$
  \https{chase.com}}{\https{amazon.com}}, and only Amazon may modify
it.\footnote{$\lor$ and $\land$ denote disjunction and conjunction. A
  comma separates the secrecy and integrity formulas.}
%
Amazon could assign this label to its order history page to allow a
Chase-hosted mashup to read the user's purchases.
%
On the other hand, after a third-party mashup hosted by
\https{mint.com} (as described in
Section~\ref{sec:motivating-examples}) reads {\em both} the user's
Chase bank statement data {\em and} Amazon purchase data, the label on
data produced by the third-party mashup will be
\dcLabel{\https{amazon.com} $\land$
  \https{chase.com}}{\https{mint.com}}.
%
This secrecy label component specifies that the data may be sensitive
to both parties, and without both their consent (see
Section~\ref{sec:system:privileges}), it should only be read by the user;
the integrity label component, on the other hand, permits only code
hosted by Mint to modify the resulting data.
 
COWL enforces label policies in a MAC fashion by only allowing a context
to communicate with other contexts or servers whose labels are at
least as restricting. (A server's ``label'' is simply its origin.)
%
Intuitively, when a context wishes to send a message, the target must
not allow additional origins to read the data (preserving secrecy).
Dually, the source context must not be writable by origins not
otherwise trusted by the target. That is, the source must be at least
as trustworthy as the target.
%
We say that such a target label ``subsumes'' the source
label.
%%\footnote{ If communication is bidirectional, \emph{e.g.,} when a
%%  script accesses the DOM in a different context (but of the same
%%  origin), then both the target and source must subsume each other.}
%
For example, a context labeled
\dcLabel{\https{amazon.com}}{\https{mint.com}} can send messages to
one labeled \dcLabel{\https{amazon.com} $\land$
\https{chase.com}}{\https{mint.com}}, since the latter is trusted
to preserve the privacy of \https{amazon.com} (and
\https{chase.com}).
%
However, communication in the reverse direction is not possible since
it may violate the privacy of \https{chase.com}.
%
In the rest of this paper, we limit our discussion to secrecy and only
comment on integrity where relevant; we refer the interested reader
to~\cite{stefan:2011:dclabels} for a full description of the label
model.

A context can freely \emph{raise} its label, \emph{i.e.,} change its label to
any label that is more restricting, in order to receive a message
from an otherwise prohibited context.
%\Red{modulo clearance?} XXX(ds): Yeah, but I don't think we should
%clutter the description with this especially since it's talked about
%at the end of the paragraph.
%
Of course, in raising its label to read more sensitive data from
another context, the context also becomes more restricted in where it
can write.
%
For example, a Mint context labeled
\dcLabelS{\https{amazon.com}}{\https{mint.com}} can raise its label to
\dcLabelS{\https{amazon.com} $\land$
  \https{chase.com}}{\https{mint.com}} to read bank statements, but
only at the cost of giving up its ability to communicate with Amazon
(or, for that matter, any other) servers.
%
When creating a new context, code can impose an upper bound on the
context's label to ensure that untrusted code cannot raise its label
and read data above this \emph{clearance}.
%
This notion of clearance is well
established~\cite{efstathopoulos:asbestos, Zeldovich:2006,
  stefan:2011:flexible, Breeze13}; we discuss its relevance to covert
channels in Section~\ref{sec:discussion}.

As noted, COWL allows a labeled context to create additional labeled contexts,
much as today's browsing contexts can create sub-compartments
in the form of iframes, workers, \emph{etc.}
%
This functionality is crucial for compartmentalizing a system
hierarchically, where the developer places code of different degrees
of trustworthiness in separate contexts.
%
For example, in the password checker example in
Section~\ref{sec:motivating-examples}, we create a child context in
which we execute the untrusted checker script.
%
Importantly, however, code should not be able to leak information by
laundering data through a newly created context.
%
Hence, a newly created context implicitly inherits the current label
of its parent.
%
Alternatively, when creating a child, the parent may specify an
initial current label for the child that is {\em more} restrictive
than the parent's, to confine the child further.
%
Top-level contexts (\emph{i.e.,} pages) are assigned a default
label of \js|public|, to ensure compatibility with pages written for
the legacy SOP.
%
Such browsing contexts can be restricted by setting a \js|COWL-label|
HTTP response header, which dictates the minimal document label the
browser must enforce on the associated content.

COWL applications can create two types of context.
%
First, an application can create standard (but labeled) contexts in
the form of pages, iframes, workers, \emph{etc.}
%
Indeed, it may do so because a COWL application is merely a regular
web application that additionally uses the COWL API\@. It thus is
confined by MAC, in addition to today's web security policies.
%
Note that to enforce MAC, COWL must mediate all pre-existing
communication channels---even subtle and implicit channels, such as
content loading---according to contexts' labels.
%
We describe how COWL does so in Section~\ref{sec:implementation}.

Second, a COWL application can create labeled contexts in the form of
\emph{lightweight labeled workers (LWorkers)}.
%
Like normal workers~\cite{workers}, the API exposed to LWorkers is
compact; it consists only of constructs for communicating with
the parent, the XHR constructor, and the COWL API\@.
%
Unlike normal workers, which execute in separate threads, an LWorker
executes in the same thread as its parent, sharing its event loop.
%
This sharing has the added benefit of allowing the parent to give the
child (labeled) access to its DOM, any access to which is treated as
both a read and a write, \emph{i.e.,} bidirectional communication.
%
Our third-party library example uses such a \emph{DOM worker} to
isolate the trusted application code, which requires access to the
DOM, from the untrusted jQuery library.
%
In general, LWorkers---especially when given DOM access---simplify the
isolation and confinement of scripts (\emph{e.g.,} the password strength
checker) that would otherwise run in a shared context, as when loaded
with \js|script| tags.

\subsection{Labeled Communication}
\label{sec:system:communication}
Since COWL enforces a label check whenever a context sends a message,
the design described thus far is already symmetric: a source context
can confine a target context by raising its label (or a child
context's label) and thereafter send the desired message. To read this
message, the target context must confine itself by raising its label
accordingly. These semantics can make interactions between contexts
cumbersome, however. For example, a sending context may wish to
communicate with multiple contexts, and need to confine those
target contexts with different labels, or even confine the same target
context with different labels for different messages. And a receiving
context may need unfettered communication with one or more origins for
a time before confining itself by raising its label to receive a
message. In the password-checker example application, the untrusted
checker script at the right of Figure~\ref{fig:checker} exhibits
exactly this latter behavior: it needs to communicate with untrusted
remote origin {\tt sketchy.ru} before reading the password labeled
{\tt fb.com}.

%% %
%% Unfortunately, this approach of (ab)using the context label to impose
%% restrictions on a target context is cumbersome.
%% %
%% (Consider the case where a context wishes to impose different
%% restrictions according to the message or target context.)

\paragraph{Labeled Blob Messages (Intra-Browser)}
To simplify communication with confinement, we introduce the {\em
  labeled Blob,} which binds together the payload of an individual
inter-context message with the label protecting it. The payload takes
the form of a serialized immutable object of type Blob~\cite{html5}.
Encapsulating the label with the message avoids the cumbersome label
raises heretofore necessary in both sending and receiving contexts
before a message may even be sent or received. Instead, COWL allows
the developer sending a message from a context to specify the label to
be attached to a labeled Blob; any label as or more restrictive than the
sending context's current label may be specified. While the receiving
context may receive a labeled Blob with no immediate effect on the
origins with which it can communicate, it may only inspect the label,
not the payload.\footnote{The label itself cannot leak
information---COWL still ensures that the target context's label is at
least as restricting as that of the source.}
Only after raising its label as needed may the receiving context read
the payload.
%
%% A labeled Blob encapsulates a serialized immutable object, of type
%% Blob~\cite{html5}, and the label protecting it.
%
%% Since messages between contexts must be serializable objects, labeled
%% Blobs can be used to associate explicit labels with messages.
%
%% Unsurprisingly, these ``labeled messages'' can be sent from one
%% context to another.
%
%% When doing so, COWL does not impose any more restrictions than those of
%% Section~\ref{sec:system:contexts}.
%% %
%% Importantly, however, a receiving context cannot arbitrarily inspect
%% such labeled messages;
%% %
%% until the message is ``unlabeled'' and the context label is raised to
%% subsume the blob label, the received labeled message is mostly opaque:
%% the receiving context can only inspect its label.

Labeled Blobs simplify building applications that incorporate distrust
among contexts. Not only can a sender impose confinement on a receiver
simply by labeling a message; a receiver can delay inspecting a
sensitive message until it has completed communication with untrusted
origins (as does the checker script in Figure~\ref{fig:checker}). They
also ease the implementation of integrity in applications, as they
allow a context that is not trusted to modify content in some other
context to serve as a passive conduit for a message from a third
context that {\em is} so trusted.
%
%% For example, a sender can impose confinement restrictions on a
%% receiver by simply labeling a message.
%
%% Since the label of the message can be above the sender's context label
%% this alleviates the need for the sender to raise its label only to
%% impose restrictions on the receiver.
%
%% (Indeed, when creating a labeled Blob, the specified label \emph{must}
%% be at least as restricting as the current context label.)
%
%% Equally important, the receiver can delay inspecting a sensitive
%% message until it is ready to raise its label and be more restricted in
%% where it can write.
%
%% (Or alternatively, it can serve as an intermediary and pass the
%% labeled Blob to yet another context, a pattern we found useful when
%% considering ``high'' integrity message.)
%
%% For example, in the password checker example of
%% Section~\ref{sec:motivating-examples} we label the password
%% \https{fb.com} before sending it to the untrusted checker.
%% %
%% The latter has the flexibility of communicating arbitrarily until it
%% unlabels the password, at which point it will be constrained to
%% sending messages to its parent.

\paragraph{Labeled XHR Messages (Browser--Server)}
\label{sec:system:labeled-xhr}
Thus far we have focused on confinement as it arises when two browser
contexts communicate. Confinement is of use in browser-server
communication, too. As noted in Section~\ref{sec:system:contexts},
COWL only allows a context to communicate with a server (whether with
XHR, retrieving an image, or otherwise) when the server's origin
subsumes the context's label. Upon receiving a request, a COWL-aware
web server may also wish to know the current label of the context that
initiated it. For this reason, COWL attaches the current label to
every request the browser sends to a server.\footnote{COWL also
  attaches the current privilege; see Section~\ref{sec:system:privileges}.}
  As also noted in
Section~\ref{sec:system:contexts}, a COWL-aware web server may elect
to label a response it sends the client by including a \js|COWL-label|
header on it. In such cases, the COWL-aware browser will only allow
the receiving context to read an XHR response if its current label
subsumes that on the response.

%% The ability of a context to impose confinement restrictions on
%% another naturally extends to servers, when using the XHR constructor.
%% % 
%% Traditionally, an XHR request is composed of two parts: a write to an
%% external origin (sending the request), and then a read from the external
%% origin (reading the response, which is restricted according to the
%% SOP).
%% %
%% Before initiating a request, COWL, however, also checks to see
%% if the label of the origin subsumes the label of context. 
%% %
%% Since COWL-enabled servers may be interested in the current label of
%% the context which initiated the request, we send this (and a
%% description of the context privileges) with every request.
%
%% Similarly, when reading the response, we check to see if the label of
%% the context subsumes the label of the response, which may be
%% explicitly set with the \js|COWL-label| header.
%% %

Here, again, a context that receives labeled data---in this case from
a server---may wish to defer raising its label until it has
completed communication with other remote origins. To give a context
this freedom, COWL supports {\em labeled XHR} communication. When a
script invokes COWL's labeled XHR constructor, COWL delivers the
response to the initiating script as a labeled Blob. Just as with
labeled Blob intra-browser IPC, the script is then free to delay
raising its label to read the payload of the response---and delay
being confined---until after it has completed its other remote
communication.
%% While the server can set the response label (with the \js|COWL-label|)
%% to ensure that the context can only inspect the response if it is
%% appropriately confined, the client can decide to delay this decision by
%% setting the response type---when opening the connection---to
%% \js|labeled-blob|.
%% %
%% As with inter-context labeled communication, the returned response
%% of this \emph{labeled XHR} constructor is a labeled Blob (whose label
%% is specified by the \js|COWL-label| header).
%% %
%% This flexibly allows the application to delay the inspection of a
%% sensitive response, which would confine it.
%
For example, in the third-party mashup example, Mint only confines
itself once it has received all necessary (labeled) responses from
both Amazon and Chase. At this point it processes the data and
displays results to the user, but it can no longer send requests since
doing so may leak information.\footnote{To continuously process data
  in ``streaming'' fashion, one may partition the application into
  contexts that poll Amazon and Chase's servers for new data and pass
  labeled responses to the confined context that processes the
  payloads of the responses.}

%% With MAC in place, we envision that there are scenarios in which web
%% servers would share resources by setting more flexible CORS policies, but
%% only do so if the client to be appropriately confined.
%% %
%% For instance, 
%
%% This is appealing because it allows a server to share resources
%% without completely trusting the white-listed cross-origin.
%
%
%% Of course, and as discussed in Section~\ref{sec:discussion}, a malicious
%% Mint application can leak data through covert channels.
%
%% Nevertheless, this improves the state of affairs where users give up
%% their credentials and place complete trust in Mint---not only to keep
%% their bank statements confidential, but also from stealing their
%% funds.


%% The operator of a server that provides useful data to mashup
%% applications cannot be expected to know all applications users may
%% wish to use. A user might prefer the niche \https{benjamins.biz}
%% application to the \https{mint.com} one. If Chase doesn't
%% CORS-whitelist the lesser-known application's origin, the user will
%% still face today's unpalatable choice between functionality and
%% privacy. Note that in a COWL-enabled browser, however, it is safe to
%% relax the CORS policy to allow a foreign origin to read Chase's data,
%% {\em so long as} Chase's responses are labeled---in which case COWL
%% will confine any script that reads them. COWL thus allows a user to
%% configure her browser to effectively add additional foreign origins to
%% those whitelisted (if any) in the CORS policy returned by an origin's
%% web server. For example, the user can configure her COWL-enabled
%% browser to allow \https{benjamins.biz} to read data from the
%% \https{chase.com} origin. If the server labels its response label,
%% COWL confines the context that reads the response accordingly.

%% Since trust varies between users, a server cannot be expected to use
%% CORS white-lists (beyond a handful of reputable domains) to
%% accommodate everybody.  For example, a user may wish to use a
%% lesser-known mashup, \https{wutang-financial.com}, instead of
%% \https{mint.com}; unless Chase white-lists this domain (maybe at the
%% user's request), the user is, again, forced to decide between giving
%% up their privacy (credentials) and using the service.  Intuitively,
%% however, it is safe relax the CORS policy to allow cross-origin code
%% to read the result of a response, as long as it is confined
%% thereafter.  Hence, COWL allows users to override the default CORS
%% policy of a website (if there is any) to white-list origins they
%% trust.  For example, the user can white-list
%% \https{wutang-financial.com} to read cross-origin data from
%% \https{chase.com}.  Importantly, however, all such cross-origin
%% responses must be labeled.  (We call this feature labeled CORS
%% (LCORS).) 
%% %
%% A server that is protected against self-exfiltration and cross-site
%% request forgery using end-to-end labels can relax this requirement by
%% supplying a less restricting label.

\subsection{Privileges}
\label{sec:system:privileges}

While confinement handily enforces secrecy, there are occasions when
an application must eschew confinement in order to achieve its goals,
and yet can uphold secrecy while doing so. For example, a context may
be confined with respect to some origin (say, \https{a.com}) as a
result of having received data from that origin, but may need to send
an encrypted version of that data to a third-party origin. Doing so
does not disclose sensitive data, but COWL would normally prohibit
such an operation. In such situations, how can a context
\emph{declassify} data, and thus be permitted to send to an arbitrary
recipient, or avoid the recipient's being confined?

COWL's {\em privilege} primitive enables safe declassification. A
context may hold one or more privileges, each with respect to some
origin. Possession of a privilege for an origin by a context denotes
trust that the scripts that execute within that context will not
compromise the secrecy of data from that origin. Where might such
trust come from (and hence how are privileges granted)? Under the SOP,
when a browser retrieves a page from \https{a.com}, any script within
the context for the page is trusted not to violate the secrecy of
\https{a.com}'s data, as these scripts are deemed to be executing on
behalf of \https{a.com}. COWL makes the analogous assumption by
granting the privilege for \https{a.com} to the context that retrieves
a from \https{a.com}: scripts executing in that context
are similarly deemed to be executing on behalf of \https{a.com}, and
thus are trusted not to leak \https{a.com}'s data to unauthorized
parties---even though they can declassify data. Only the COWL runtime
can create a new privilege for a valid remote origin upon
retrieval of a page from that origin; a script cannot synthesize a
privilege for a valid remote origin.

To illustrate the role of privileges in declassification, consider the
encrypted Google Docs example application. In the implementation
of this application atop COWL, code executing on behalf of
\https{eff.org} (\emph{i.e.,} in a compartment holding the \https{eff.org}
privilege) with a current label \dcLabelS{\https{eff.org} $\land$
  \https{gdoc.com}}{} is permitted to send messages to a context
labeled \dcLabelS{\https{gdoc.com}}{}.
%
Without the \https{eff.org} privilege, this flow would not be allowed,
as it may leak the EFF's information to Google.
%
%% EFF is declassifying the message by using privileges. \emph{i.e.,} it
%% no longer considers it sensitive (since, for instance, it may be
%% encrypted).

Similarly, code can declassify information when unlabeling messages.
%
Consider now the password checker example application. The left
context in Figure~\ref{fig:checker} leverages its \https{fb.com}
privilege to declassify the password strength result, which is labeled
with its origin, to avoid (unecessarily) raising its label to
\https{fb.com}.
%
%% (Alternatively it would need to use the privilege to lower its current
%% label.)


% COWL analog: when browser retrieves page from a.com, the context in
% which the page resides is trusted not to violate secrecy of a.com's
% data. COWL renders this trust explicit by granting this context the
% \emph{privilege} for a.com. Scripts executing in that context are
% deemed to be executing on behalf of a.com. They are therefore
% trusted to perform two operations that would otherwise be vulnerable
% to abuse by a potentially malicious script executing on behalf of an
% untrusted principal: declassification and delegation.

% Declassification denotes stripping a secrecy label from data, and
% thus avoiding confinement of the data's recipient. Intuitively, if a
% context has the privilege for an origin and scripts executing within
% that context are thus trusted to be executing on behalf of that
% origin, then they can be trusted not to leak sensitive information.
% In some cases, a context within an application must declassify in
% order to achieve the desired functionality. [gdocs example]

% COWL exercises privileges implicitly upon message receipt (both of
% postMessage messages and labeled Blobs). That is,
% if a compartment holds the privilege for an origin, it always
% declassifies data labeled with that origin upon receipt---\emph{i.e.,} it
% need not raise its label to include that origin in order to read the
% message's payload. COWL eschews implicit declassification on
% sending messages, however, as errors of omission would then
% result in 

% A context that holds the privilege for
% a.com declassifies data with respect to a.com before sending
% it---that is, the receiving context need not raise its label to
% include a.com to receive the data, and thus is not confined by
% a.com. Because holding the privilege for a.com denotes being trusted
% not to violate the secrecy of a.com's data, COWL permits this
% declassification. 


%% COWL provides every context with unforgeable objects called
%% \emph{privileges} with which code can assert authority over
%% origins.
%% %
%% Privileges are used to allow flows otherwise not permitted by COWL's
%% label checks.
%
%

COWL generally exercises privileges \emph{implicitly:} if a context
holds a privilege, code executing in that context will, with the
exception of sending a message, always attempt to use it.\footnote{
While the alternative approach of explicit exercise of privileges
(\emph{e.g.,} when registering an \texttt{onmessage} handler) may be
safer~\cite{Zeldovich:2006, stefan:2011:flexible, miller:capability},
we find it a poor fit with 
existing asynchronous web APIs.}
%% \footnote{ Exposing a new set of communication primitives would
%%   simplify this, but one our goals with COWL is to avoid imposing new
%%   programming models on developers.  }
%
COWL, however, lets code control the use of privileges by allowing code
to get and set the underlying context's privileges.
%
Code can drop privileges by setting its context's privileges to
\js|null|.
%
Dropping privileges is of practical use in confining closely
coupled untrusted libraries like jQuery.
%
Setting privileges, on the other hand, increases the
trust placed in a context by authorizing it act on behalf of
origins.
%
This is especially useful since
COWL allows one context to {\em delegate} its privileges (or a subset
of them) to another; this functionality is also instrumental in
confining untrusted libraries like jQuery.
%
%%XXX(ds): this may be relevant, but maybe too focused on
%% implementation vs. design:
%%(Since COWL is applied atop the SOP, delegating a privilege to a
%%cross-origin context will not give it the authority to, for instance,
%%bypass the SOP and access the DOM of the delegating context.)
%
Finally, COWL also allows a context to create privileges for \emph{fresh}
origins, \emph{i.e.,} unique origins that do not have a real protocol
(and thus do not map to real servers).
%
These fresh origins are primarily used to \emph{completely} confine a
context: the sender can label messages with such an origin, which upon
inspection will raise the receiver's label to this ``fake'' origin,
thereby ensuring that it cannot communicate except with the parent
(which holds the fresh origin's privilege).
%
%

%% %%XXX(ds): MOVE to implementation:
%% For example, our implementation modifies existing browsing contexts to
%% mediate:
%% %
%% \js|postMessage|,
%% same-origin cross-compartment DOM object access,
%% content loading (via {\tt img} tags, CSS, \emph{etc.}),
%% form submission,
%% use of the \js|XMLHttpRequest| object,
%% navigation, and
%% browser storage (cookies and local storage).



% Local Variables:
% TeX-master: "main.ltx"
% TeX-command-default: "Make"
% End:

