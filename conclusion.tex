\section{Conclusion}
\label{sec:conclusion}

Web applications routinely pull together JavaScript contributed by
parties untrusted by the user, as well as by mutually distrusting
parties.
% And they routinely compute over users' sensitive data.
The lack of robust confinement for untrusted code in the status-quo
browser security architecture puts users' privacy at risk.
% : untrusted
% third-party libraries can leak sensitive data from the applications
% that include them, and third-party mashup developers resort to
% demanding users' login credentials for other online services whose
% content they wish to integrate.
In this paper, we have presented
\sys{}, a label-based MAC system for web browsers that preserves
users' privacy in the common case where untrusted code computes over
sensitive data. \sys{} affords developers flexibility in synthesizing
web applications out of untrusted code and services while preserving
users' privacy. Our positive experience building four web applications
atop \sys{} for which privacy had previously been unattainable in
status-quo web browsers suggests that \sys{} holds promise as a
practical platform for preserving privacy in today's pastiche-like web
applications. And our measurements of \sys{}'s performance overhead in
the Firefox and Chromium browsers suggest that \sys{}'s privacy
benefits come at negligible end-to-end cost in performance.

% \Red{Future work: phishing, IFC loading, making it easier for developers to privilege separate, doing better, more realistic applications}
