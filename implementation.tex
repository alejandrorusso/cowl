\section{Implementation}
\label{sec:implementation}

We implemented \sys{} in the Firefox and Chrome browsers.
%
The API shown in Table~\toref{table:components} is specified in 264
lines of WebIDL~\cite{webidl}, common to both
implementations.
%
The remaining system is implemented in C++, modifying/extending the
Gecko and Blink layout engines.
%
In all our line-of-code (LOC) reports, we simply count the length of
the source files which includes comments and empty-lines.
%
Below we detail the implementation of the various \sys{} components;
in Section~\ref{sec:eval}, we describe the implementation of the
Section~\ref{sec:system} applications.

\subsection{Policies}
%
Our implementation of \js|Label|s and \js|Privilege|s is simply a port
of the Haskell implementation presented
in~\cite{stefan:2011:dclabels,stefan:2011:flexible}. 
%
This implementation is straight forward and only differs in the two
browsers because of their dependence on different container libraries;
the Firefox and Chrome implementations are roughly 1K and \Red{XXX}
lines of C++ code, respectively.
%
We note that much of this is a result of using C++ (and thus having
to, for example, implement method overloading to support a friendly
API)---as a comparison our JavaScript implementation of labels is
roughly 100 lines.
%

\subsubsection{Firefox}

Gecko's isolation model relies on \emph{compartments}, i.e., disjoint
JavaScript heaps, for both garbage collection (GC) and
security~\tocite{franz}.
%
The isolation is guaranteed by ensuring that all cross-compartment
communication (e.g., \js|postMessage| between iframes) is done through
\emph{wrappers}---an object from one compartment can never reference
another object from a different compartment directly.
%
This has the benefit that GC can be done in parallel, on different
compartments, and that all inter-browsing context access control
checks---as specified by the SOP---can be enforced by the wrappers.
%
Naturally, this is possible because each compartment has a security
principal, i.e., the origin, which is used in making these policy
decision, in addition to those described in
Section~\ref{sec:background} (e.g., SOP, CSP and CORS when using XHR).

\paragraph{Chrome}

\subsection{Firefox}
\label{sec:implementation:firefox}


