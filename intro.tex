\section{Introduction}
\label{sec:intro}

Web applications have proliferated because it is so easy for
developers to reuse components of existing ones. While component reuse
in the venerable desktop software model typically occurs for libraries,
the reusable components in web applications are not limited to just
JavaScript library code---they further include network-accessible
content and services. Such reuse is ubiquitous\@. jQuery, a widely used
JavaScript library, is included in and used by over 77\% of the
Quantcast top-10,000 web sites, and 59\% of the Quantcast top-million
web sites~\cite{quantcast}.

The resulting model is one in which web developers cobble together
multiple JavaScript libraries, web-based content, and web-based
services written and operated by various parties (who in turn may
integrate more of these resources) and build the required
application-specific functionality atop them. Unfortunately, some of
the many contributors to the tangle of JavaScript comprising an
application may not have the user's best interest at heart. The wealth
of sensitive data processed in today's web applications (e.g., email,
bank statements, health records, passwords, etc.) is an attractive
target. Miscreants may stealthily craft malicious JavaScript that,
when incorporated into an application by an unwitting developer,
violates the user's privacy by leaking sensitive information.

Two goals for web applications emerge from the prior discussion:
\emph{flexibility} for the application developer (i.e., enabling the
building of applications with rich functionality, composable from
potentially disparate pieces hosted by different sites); and
\emph{privacy} for the user (i.e., to ensure that the user's sensitive
data cannot be leaked from applications to unauthorized parties).
These two goals are hardly new: Wang {\em et al.}~articulated similar
ones, and proposed new browser primitives to improve isolation within
{\em mashups,} including discretionary access control (DAC) for inter-frame
communication~\cite{mashupos}. Indeed, today's browsers incorporate
similar mechanisms in the guises of HTML5's seamless iframes, iframe
sandbox, and postMessage API~\cite{html5}. And the Same-Origin Policy
(SOP, reviewed in Section~\ref{sec:backgd}) prevents JavaScript hosted
by one principal from reading content hosted by another.

Unfortunately, in the status-quo web browser security architecture,
one must often sacrifice privacy to achieve flexibility, and
vice-versa. The central reason that flexibility and privacy are at
odds in the status quo \todo{}{Consistency: status quo or status-quo} is that the mechanisms today's browsers rely on
for providing privacy---the SOP, Content Security Policy
(CSP)~\cite{csp}, and Cross-Origin Resource Sharing
(CORS)~\cite{cors13}---are all forms of discretionary access control.
DAC has the brittle character of either denying or granting
untrusted code (e.g., a library written by a third party) access to
data. In the former case, the untrusted JavaScript might {\em need}
the sensitive data to implement the desired application
functionality---hence, denying access prioritizes privacy over
flexibility. In the latter, DAC exercises no control over what the
untrusted code does with the sensitive data---and thus prioritizes
flexibility over privacy. DAC is an essential tool in the privacy
arsenal, but {\em does not fit cases where one runs untrusted code on
  sensitive input,} which are the norm for web applications, given
their multi-contributor nature.

In practice, web developers turn their backs on privacy in favor of
flexibility because the browser doesn't offer primitives that let them
opt for both. For example, a developer may want to include untrusted
JavaScript from another origin in his application. All-or-nothing DAC
leads the developer to include the untrusted library with a
\js|<script>| tag, which effectively bypasses the SOP, interpolating
untrusted code into the enclosing page and granting it unfettered
access to the enclosing page's origin's content.\footnote{Indeed,
  jQuery \emph{requires} such access to the enclosing page's content!}
And when a developer of a mashup that integrates content from two {\em
  other} origins finds that the SOP forbids his application from
retrieving data from them, he designs his mashup to require that the
user provide the mashup her login credentials for the sites at the two
other origins~\cite{mint.com}---the epitome of ``functionality over
privacy.''

In this paper, we present \sys{} (Confinement with Origin Web Labels),
a mandatory access control (MAC) system that confines untrusted
Javascript in web browsers. \sys{} allows untrusted code to compute
over sensitive data and display results to the user, but prohibits the
untrusted code from exfiltrating sensitive data (e.g., by sending it
to an untrusted remote origin). It thus allows web developers to opt
for {\em both} flexibility and privacy.

We consider four motivating example web ap\-pli\-ca\-tions---a
password strength-checker, an application that imports the (untrusted)
jQuery library, an encrypted cloud-based document editor, and a
third-party mashup, none of which can be implemented in a way that
preserves the user's privacy in the status-quo web security
architecture. These examples drive the design requirements for \sys{},
particularly MAC with \emph{symmetric and hierarchical confinement} and
{\em delegation.}  Symmetric confinement allows \emph{mutually}
distrusting principals each to pass sensitive data to the other, and
confine the other's use of the passed sensitive data. Hierarchical
confinement allows any developer to confine code she does not trust,
and confinement to be nested to arbitrary depths. And delegation
allows a developer explicitly to confer the privileges of one execution
context explicitly on a separate execution context.
%
No prior browser security architecture offers this combination of
properties.

We demonstrate the applicability of \sys{} by using it to implement
secure versions of the four motivating applications.
Our contributions include:

\begin{CompactItemize}
\item We characterize the shared needs of four case-study web applications
  (Section~\ref{sec:motivating-examples}) for which today's browser
  security architecture cannot provide privacy.
%strong 
\item We describe the design of the \sys{} label-based MAC system
  for web browsers (Section~\ref{sec:system}), which meets the
  requirements of the four case-study web applications.
\item We describe designs of the four case-study web applications atop
  \sys~(Section~\ref{sec:apps}).
\item We describe implementations of \sys{}
  (Section~\ref{sec:implementation}) for the Firefox and Chromium
  open-source browsers; our evaluation (Section~\ref{sec:eval})
  illustrates that \sys{} incurs minimal performance overhead over the
  respective baseline browsers.
\end{CompactItemize}

% Local Variables:
% TeX-master: "main.ltx"
% TeX-command-default: "Make"
% End:
