\section{Evaluation}
\label{sec:eval}

In this section we describe the performance of evaluation of \sys{}.
%
We first evaluate the end-to-end performance of the \sys{}
applications described in Section~\ref{sec:system}, relative to
implementations that simply trade off security for the features the
apps provide.
%
Then, we measure the performance impact of \sys{} on existing browser
features and the additional overhead when confinement-mode is actually
enabled.
5 
All our measurements were conducted on a 4-core i7-2620M machine
running GNU/Linux 3.11, with 16GB of RAM.
%
Our patches are applied to Firefox versions 26 and Chromium version
31.
%
%% Below, we use the \emph{vanilla} to denote unmodified browser;
%% \emph{unlabeled} to denote a modified browser with confinement-mode
%% off; and, \emph{labeled} to denote a modified browser with \sys{} in
%% active use.


\subsection{Macro benchmarks}
\label{sec:eval:macro}

\paragraph{Password-strength checker}
%
We measured the time it take to check the strength of a randomly generated
16-character password.
%

Vanilla: just run script in worker

Firefox: With \sys{} it take 59.5ms, Vanilla: 61.4
Chome: With \sys{} it take 80.5ms, Vanilla:  71.5


\paragraph{Encrypted cloud-based editor}
%
we measured the time it takes to save an encrypted a document
\todo{XXX}-character document.

Vanilla: removed the sandbox stuff.

Firefox: With \sys{} it take 114.8s, Vanilla: 100.2
Chome: With \sys{} it take 248.1, Vanilla; 257.8s

\paragraph{Third-party mashup}

We measured the time it takes to perform requests to two different origins,
which return a simple \todo{XXX}-JSON object, including a number.
%
Our mashup benchmark simply adds the numbers.


Firefox: With \sys{} it take 10.7ms, Vanilla:  10.7ms
Chrome: With \sys{} it take 15ms, Vanilla:  17ms

Vanilla: just remove sandbox stuff + add CORS

\paragraph{Phone2Links}

We measure the load time of a bank page incorporating the Phone2Links library;
in the first case, the bank trusted code is placed a LWorker, following the
pattern of Section~\ref{sec:system:script}; in the second case, the untrusted
code is treated like an extension and placed in an LWorker with no privileges,
following the pattern of Section~\ref{sec:system:extension}.

Vanilla: incorporate script directly, no sandboxes.


As page confine:
Firefox: With \sys{} it take 164.8ms, Vanilla:  160.2ms

As extension
Firefox: With \sys{} it take 224.4ms

\subsection{Micro benchmarks}
\label{sec:eval:micro}

\newcommand*\rot{\rotatebox{90}}

\begin{table}
\centering
\begin{tabular}{l |c|c|c|c|c|c }
\toprule
                   & \multicolumn{3}{c}{\textbf{Firefox}}
                   & \multicolumn{3}{c}{\textbf{Chrome}} \\
                   & \rot{vanilla}   &
                     \rot{unlabeled} &
                     \rot{labeled}   &
                     \rot{vanilla}   &
                     \rot{unlabeled} &
                     \rot{labeled}   
\\\midrule%--------------------------------------------------------------------
% cost of creating an iframe                                         
New iframe         &   13.27  &  13.3   & 13.97   &   50.63 &   48.73 &  51.77 
\\\hline%----------------------------------------------------------------------
% cost of creating a worker                                          
New worker         &  20.53  &   21.24 &  1.62   &  18.94  &  18.87  & 3.27
\\\midrule%--------------------------------------------------------------------
% cost of labeled postMessage                                        
Iframe ping-pong   &  0.086  &  0.084  &  0.092  &  0.042  &  0.042  &  0.042
\\\hline%----------------------------------------------------------------------
% cost of performing XHR                                             
XHR ping-pong      &  3.33   &   3.48  &  3.49   &  7.0    &  7.4    & 7.2
\\\hline%----------------------------------------------------------------------
% cost of worker ping-pong                                           
Worker ping-pong   &  0.059  &   0.055 &  0.037  &  0.071  &  0.070  & 0.030
\\\midrule%--------------------------------------------------------------------
% cost of setting current label                                      
Set label          &  --     &  \multicolumn{2}{c|}{0.0014} &   --   
                             &  \multicolumn{2}{c }{0.0018}
\\\hline%-------------------------------------------------------------------------------------------------
% cost of setting current label                                                                   
Get label          &  --     &  \multicolumn{2}{c|}{0.0003} &   --   
                             &  \multicolumn{2}{c }{0.0008}
\\\hline%-------------------------------------------------------------------------------------------------
% cost of comparing labels                                                                        
Label subsumes     &  --     &  \multicolumn{2}{c|}{0.0012} &   --   
                             &  \multicolumn{2}{c }{0.0004}
\\\bottomrule
\end{tabular}
\caption{\label{microbench} Micro-benchmarks (in milliseconds).
%%Vanilla denotes stock browser measurements, our modified browsers with
%%confinement-mode turned off and on are respectively
%%are denoted by unlabeled and labeled.
}
\end{table}

Table~\ref{microbench} shows our micro-benchmarks for the stock
browsers (vanilla), our modified browsers with confinement-mode turned
off (unlabeled), and confinement-mode enabled (labeled).
%
We measure the cost of creating various compartments (workers and
iframes), sending messages between various entities
(compartments and network), and common label operations.

The performance impact of creating an iframe or worker is
within 3.5\% and 2.5\% of the vanilla implementations on Firefox and
Chromium, respectively.
%
In absolute terms, \sys{} adds, on average, less than 1 millisecond
latency.
%
As also shown in the table, the \sys{} labeled workers provide
programmers with a considerably more light weight concurrency
primitive: creating a labeled worker is roughly 12 and 6 times faster
than a corresponding unlabeled worker.

%
The iframe, worker and XHR ping-pong measurements evaluate the cost of 
sending a message across iframes, workers, and network, respectively.
%
In all cases, the performance impact of \sys{} is negligible when
compared to an unmodified, vanilla browser.
%
In the case of XHR, the performance impact on the
(local-network test) is less than 5.5\% for both Firefox and Chromium,
a less than 0.5 millisecond delay.
%
In most of the other cases the sub 0.1 millisecond differences can be
attributed to noise.
%
The only interesting case is that of communication with
labeled-workers; as with creating labeled workers, communication with
a labeled workers is roughly 2 times faster than a corresponding
vanilla worker.

Finally, getting the current label, setting the current label and
performing a label comparison (subsumes) are all on the order of a
microsecond.
%
These measurements, as in the case of sending labeled messages, use
a single label: the public label.
%
Hence, when considering more complex labels, we expect the performance
impact of labels to naturally be higher.
%
For a bounded random label comparison, with at most 5 principals, we
measured a 5\% performance degradation.
%
This measurement serves as a reasonable point for real application,
which are likely to only contain a handful of principals.
%
While the 5\% slowdown (for a rare operation) is mostly negligible, we
note there is much room for speeding up label comparisons since our
label implementation has not yet been optimized.

%%Comparing the 
%%Similarly, the performance impact of labels is under  5\%  in almost
%%every case.

