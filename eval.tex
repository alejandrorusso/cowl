\section{Evaluation}
\label{sec:eval}

Performance largely determines acceptance of new browser features
in practice.
%
To evaluate the performance of \sys{}, we ran experiments to quantify
the cost of our new primitives, as well as their impact on legacy web
sites that do not use \sys{}'s features.
%
Our experiments consist of micro-benchmarks on API functions and
end-to-end benchmarks of our example applications.
%
We conducted our measurements on a 4-core i7-2620M machine with 16GB
of RAM\@ running GNU/Linux 3.13; we used the Node.js web server
over a loopback interface for all our applications.

\subsection{Micro-Benchmarks}
\label{sec:eval:micro}

\newcommand*\rot{\rotatebox{90}}

\begin{table}
\centering
\resizebox{1.0\columnwidth}{!}{
\begin{tabular}{l |c|c|c|c|c|c }
\toprule
                   & \multicolumn{3}{c}{\textbf{Firefox}}
                   & \multicolumn{3}{c}{\textbf{Chromium}} \\
                   & \rot{vanilla}   &
                     \rot{unlabeled} &
                     \rot{labeled}   &
                     \rot{vanilla}   &
                     \rot{unlabeled} &
                     \rot{labeled}   
\\\midrule%--------------------------------------------------------------------
% cost of creating an iframe                                         
New iframe         & 14.4    &  14.5    &  14.4   &   50.6 &   48.7 & 51.8  %% avg of 10 runs
\\
% cost of creating a worker                                          
New worker         & 15.9    &  15.4   &   0.90$\dagger$    &  18.9  &  18.9  & 3.3$\dagger$  %% avg of 10 runs
\\\midrule%--------------------------------------------------------------------
% cost of labeled postMessage                                        
Iframe comm.       &  0.11   &  0.11   &   0.12  &  0.04  &  0.04  & 0.04 %% avg # messages/10seconds
\\
% cost of performing XHR                                             
XHR comm           &  3.5   &   3.6   &  3.7   &  7.0    &  7.4    & 7.2 %1000request in 10 iframes
\\
% cost of worker ping-pong                                           
Worker comm.       &  0.20    &   0.24  &  0.03$\ddagger$   &  0.07  &  0.07  & 0.03$\ddagger$  % 90 messages ping-pong
%%\\\midrule%--------------------------------------------------------------------
%%% cost of setting current label                                      
%%Set label          &  --     &  \multicolumn{2}{c|}{0.0016} &   --   
%%                             &  \multicolumn{2}{c }{0.0018}
%%%10 iframes doing 10,000 ops
%%\\\hline%-------------------------------------------------------------------------------------------------
%%% cost of setting current label                                                                   
%%Get label          &  --     &  \multicolumn{2}{c|}{0.0001} &   --   
%%                             &  \multicolumn{2}{c }{0.0008}
%%\\\hline%-------------------------------------------------------------------------------------------------
%%% cost of comparing labels                                                                        
%%Label check        &  --     &  \multicolumn{2}{c|}{0.0011} &   --   
%%                             &  \multicolumn{2}{c }{0.0004}
%%%10 iframes doing 10,000 subsumes of random labels with 4 principals
\\\bottomrule
\end{tabular}
}
\caption{Micro-benchmarks, in milliseconds.}
\label{microbench} 
\end{table}

\paragraph{Context Creation}
Table~\ref{microbench} shows micro-benchmarks for the stock
browsers (vanilla), the \sys{} browsers with confinement mode turned
off (unlabeled), and with confinement mode enabled (labeled).
%
\sys{} adds negligible latency (less than 1ms) to compartment
creation.
%
We omit the measurements for labeled ``normal'' Workers since they are
the same as those for unlabeled Workers.
%
We attribute the iframe-creation speedup for Chromium to noise.
%
More interestingly, we highlight that the cost of creating LWorkers is
considerably smaller than that for ``normal'' Workers, which run on
separate OS threads ($\dagger$).

\paragraph{Communication} The iframe, worker, and XHR communication measurements evaluate the
round-trip latencies across iframes, workers, and the network.
%
For the XHR benchmark, we report the cost of using the labeled XHR
constructor averaged over 10,000 requests.
%
Our Chromium implementation uses an  LWorker to wrap the unmodified
XHR constructor, so the cost of labeled XHR incorporates an additional
cross-context call.
%
As with creation, communicating with LWorkers ($\ddagger$)
is considerably faster than for ``normal'' Workers.
%
This speedup arises because a lightweight worker shares
 an OS thread and event loop with their parent.
%
%% However, this sharing means that at 90 messages, we exhaust the call
%% stack.

%%As shown in the table, we measured the cost of getting the current
%%label, setting the current label and performing a label comparison
%%(subsumes) over the span of 10,000 operations.
%%%
%%The first two measurements, as in the case of sending labeled
%%messages, use a single label: the public label.  Since the subsumes
%%algorithm varies according to the labels, our measurements use
%%randomly generated labels, with up to 5 origins.
%%%
%%When considering labels with more principals, we expect the
%%performance impact of labels to naturally be higher; we used 5 origins
%%since we expect real apps to only contain a handful of principals.
%%%
%%The cost of all these ``core'' label operations are within
%%1 microsecond.

\paragraph{Labels} We measured the cost of setting/getting the current
label and the average cost of a label check, on Firefox.
%
For a randomly generated label with a handful of origins, these
operations take on the order of one microsecond.
%
The primary cost is recomputing cross-compartment wrappers and
the underlying CSP policy, which ends up costing up to 13ms (\emph{e.g.,} when
the label is raised from public to a third-party origin).
%
For many real applications, we expect raising the current label
to be a rare occurrence.
%
Moreover, there is much room for optimization (\emph{e.g.,}
porting COWL newest version of the CSP, which for
setting policies is $15\times$ faster~\cite{faster-csp}).
%%{\em e.g.,} in some cases,
%%we could avoid changing the underlying CSP policy and document
%%origin upon a label change when a compartment is already completely confined.
%%%%\todo{}{What about examplifying like ``, \emph{i.e.,} when the CSP policy has been set to default-src `none`''}.
 
\paragraph{DOM} We also executed the Dromaeo benchmark suite~\cite{dromaeo},
which evaluates the performance of core functionality such as
querying, traversing, and manipulating the DOM, on Firefox and
Chromium. We compared the performance of the vanilla and unlabeled
browsers to be on par: the greatest slowdown was under 4\%.

\subsection{End-to-End Benchmarks}
\label{sec:eval:macro}

To assess \sys{}'s impact on end-to-end performance, we measured
page load times for simplified versions of our example applications.
%
To focus on measuring \sys{}'s overhead, we compare our secure apps
against similarly-compartmentalized, but insecure, apps---\emph{i.e.,}
apps that perform no security checks.
%
%We acknowledge that compartmentalization

\paragraph{Password-Strength Checker}
%% Vanilla: just run script in worker
%% Firefox: With \sys{} it take 59.5ms, Vanilla: 61.4
%% Chome: With \sys{} it take 80.5ms, Vanilla:  71.5
%
%%We built the password-strength checker of
%%Section~\ref{sec:motivating-examples} using the checker script
%%from~\cite{checker1}.
%
We measured the average duration of creating a new worker, fetching a
8KB checker script (modified~\ref{sec:motivating-examples}), and
checking a password sixteen characters in length.
%
The checker takes an average of 18 ms (averaged over ten runs) on
Firefox (labeled), 4 ms less than using a Worker on vanilla Firefox.
%
Similarly, the checker running on labeled Chromium is 5ms faster than
the vanilla counterpart (measured at 54 ms).
%
In both cases the speedup is due to the fact that our LWorkers are
cheaper than normal Workers.
%
However, these measurements are roughly 5 ms slower than simply loading
the checker using an unsafe \js|script| tag.

\paragraph{Encrypted Document Editor}
%% Vanilla: removed the sandbox stuff.
%% Firefox: With \sys{} it take 114.8, Vanilla: 100.2
%% Chome: With \sys{} it take 248.1, Vanilla; 257.8

%
We measured the end-to-end time taken to load the application and
encrypt a 4KB document using the SJCL AES-128 library~\cite{sjcl}.
%
The total run time includes the time taken to load the document editor
page, which in turn loads the encryption-layer iframe, which further
loads the editor proper.
%
On Firefox (labeled) the workload completes in 116 ms;  on vanilla
Firefox, a simplified and unconfined version completes in 100ms.
%
On Chromium, the performance measurements were comparable; the
completion time was within 1ms of 244ms.
%
The most expensive operation in the COWL-enabled Firefox app is
raising the current label, since it requires changing the underlying
document origin and recomputing the cross-compartment wrappers and CSP
(to ensure that DOM and network access is restricted according to
confinement).


\paragraph{Third-Party Mashup}
%% Firefox: With \sys{} it take 10.7ms, Vanilla:  10.7ms
%% Chrome: With \sys{} it take 15ms, Vanilla:  17ms
%% 
%% Vanilla: just remove sandbox stuff + add CORS

We implemented a very simple third-party mashup application that makes
a labeled XHR request to two unaffiliated origins, each of which produces a
response containing a 27-byte JSON object with a numerical property,
and sums the responses together.
%
The corresponding vanilla app is identical, but uses the normal XHR
object.
%
In both cases we use CORS to permit cross-origin access.
%
The Firefox (labeled) workload completed in 41 ms, which is 6ms slower
than the vanilla version.
%
As in the document editor the slowdown derives from raising the
current label, though in this, case only for a single iframe.
%
On Chromium (labeled), the workload completed in 55 ms, 2 ms
slower than the vanilla one; the main slowdown here is due to our
implementation of labeled XHR using a wrapping LWorker.

\paragraph{Untrusted Third-Party Library}
%% Vanilla: incorporate script directly, no sandboxes.
%% As page confine:
%% Firefox: With \sys{} it take 164.8ms, Vanilla:  160.2ms

We measured the load time of a banking application that incorporates
jQuery and an untrusted library that traverses the DOM to replace
phone numbers with links.
%
The latter library uses XHR in attempt to leak the page's content.
%
We compartmentalize the main page into a public outer component and a
sensitive iframe containing the bank statement. In both
compartments, we place the bank's trusted code (which loads the
libraries) in a trusted labeled DOM worker with access to the page's
DOM. We treat the rest of the code as untrusted.
%
As our current Chromium implementation does not yet support DOM access
for lightweight workers, we only report measurements of Firefox.
%
The measured latency on Firefox (labeld) was 164.8 ms, a 4.6 ms
slowdown when compared to the unconfined version running on vanilla
Firefox.
%
%
This slowdown is negligible considering that the application contains
mixed-sensitivity content which COWL prevents from being exfiltrated.

%% Vanilla: incorporate script directly, no sandboxes.
%% 
%% As page confine:
%% Firefox: With \sys{} it take 164.8ms, Vanilla:  160.2ms
%% 
%% As extension
%% Firefox: With \sys{} it take 224.4ms



%%Comparing the 
%%Similarly, the performance impact of labels is under  5\%  in almost
%%every case.

