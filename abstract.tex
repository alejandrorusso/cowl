\begin{abstract}

Modern web applications comprise a conglomeration of JavaScript from
multiple authors:  third-party libraries included by a site's
developer, site-specific scripts by the site developer herself, and
third-party extensions installed in the browser by the user.  Recent
years have seen the continual discovery of practical attacks on web
users' privacy---from the leaking of sensitive data within pages by
malicious third-party library code, to similar leaks by malicious
browser extensions, to more subtle leaks, such as those via image
resources. Fundamentally, these privacy violations occur because
today's web browsers lack sufficient mechanisms for {\em confining
  untrusted code.}  We present \sys, a simple but powerful approach to
robust confinement of JavaScript in modern web browsers.
\sys\ prevents malicious third-party libraries from violating users'
privacy.  It provides safety to Mashup web applications that
previously posed an inherent risk to user data confidentiality.
\sys's flexible confinement mechanisms furthermore obviate much of the
need for privilege in browser extensions, permitting many of today's
extensions to be realized instead as untrusted web pages.  \sys\ has
been implemented in both Firefox and Chrome; measurements of both
browsers demonstrate a virtually imperceptible increase in page-load
latency.

%% - privacy vulnerabilities, lack of flexibility in mashups
%% - SOP, CORS, CSP: a patchwork of ad hoc mechanisms
%% - \sys, a simple architecture for confining JavaScript code by limiting
%%   its privileges.
%% - implementations for both Firefox and Chrome open-source browsers.
%% - new portable web applications built on \sys that demonstrate how
%%   \sys simplifies the writing of web applications that robustly preserve
%%   the user's privacy
%% - Measurements of implementations of \sys for both the Firefox
%%   and Chromium open-source browsers demonstrate that \sys's privacy
%%   guarantees incur virtually imperceptible added page-load latency.

\end{abstract}

