\begin{abstract}

% ezyang: I guess maybe we should give a teensy bit of background
% here, but not an entire paragraph's worth.
    %
%Modern web applications comprise a conglomeration of JavaScript from
%multiple authors:  third-party libraries included by a site's
%developer, site-specific scripts by the site developer herself, and
%third-party extensions installed in the browser by the user.  Recent
%years have seen the continual discovery of practical attacks on web
%users' privacy---from the leaking of sensitive data within pages by
%malicious third-party library code, to similar leaks by malicious
%browser extensions, to more subtle leaks, such as those via image
%resources. Fundamentally, these privacy violations occur because
%today's web browsers lack sufficient mechanisms for {\em confining
%  untrusted code.}

Modern web applications are conglomerations of JavaScript written by
multiple authors: application developers routinely incorporate code
from third-party libraries, and \emph{mashup} applications
synthesize data and code hosted at different sites.
% Unfortunately,
% today's browser security model sacrifices the user's privacy in many
% such applications.
In current browsers, a web application's
developer and user must trust third-party code in libraries not 
to leak the user's sensitive information from within
applications. Even worse, in the status quo, the only way to implement
some mashups is for the user to give her login credentials 
for one site to the operator of another site. Fundamentally, today's
browser security model
% too many adverbs
%effectively
trades privacy for flexibility because 
it lacks a sufficient mechanism for \emph{confining untrusted code.}
% All of these scripts have full access to
% all of a user's private data, meaning that an application must place
% full trust on every library it uses.  The ability to confine JavaScript
% in a web browser would permit applications to reduce their trust on
% libraries, while at the same time enabling mash-up web applications that
% could not previously be implemented without compromising user data
% confidentiality.
%
We present \sys, a robust JavaScript confinement system for modern web
browsers. \sys{} introduces label-based mandatory access control to
browsing contexts in a way that is fully backward-compatible with
legacy web content. We use a series of case-study applications to
motivate \sys's design and demonstrate how \sys{} allows both the
inclusion of untrusted scripts in applications and the building of
mashups that combine sensitive information from multiple mutually
distrusting origins, all while protecting users' privacy.
%
Measurements of two \sys{} implementations, one in Firefox and one in
Chromium, demonstrate a virtually imperceptible increase in page-load
latency.

%% - privacy vulnerabilities, lack of flexibility in mashups
%% - SOP, CORS, CSP: a patchwork of ad hoc mechanisms
%% - \sys, a simple architecture for confining JavaScript code by limiting
%%   its privileges.
%% - implementations for both Firefox and Chrome open-source browsers.
%% - new portable web applications built on \sys that demonstrate how
%%   \sys simplifies the writing of web applications that robustly preserve
%%   the user's privacy
%% - Measurements of implementations of \sys for both the Firefox
%%   and Chromium open-source browsers demonstrate that \sys's privacy
%%   guarantees incur virtually imperceptible added page-load latency.

\end{abstract}

