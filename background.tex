\section{Background}
\label{sec:background}



\subsection {Same origin policy}

The same origin policy is an important security concept for a number of
browser-side programming languages~\cite{TODO}. This policy aims to maintain a strict
separation between content provided by unrelated sites. It allows scripts
running on pages originating from the same web site\footnote{An origin is
  defined as a combination of scheme, hostname, and port number} to obtain data
by \js|XMLHttpRequests|. Additionally, scripts from the same origin are allowed to
access each other's DOM with no specific restrictions, but it prevents access to
DOM on scripts from different sites. 

Despite SOP, scripts can still achieved cross-origin communication in several
manners. Modern browsers impose no restrictions on how URLs are used. Scripts
are thus allowed to embedded images, scripts, styles and frames from arbitrary
domains, where the URLs used for that purpose encode information from the
current web page. Moreover, the SOP does not forbid performing cross-origin
requests (only to observation of the responses): thus, it is trivial to write
data to a third-party website. Finally, in the browser, cross-origin windows and
frames can freely communicate bi-directionally by the use of fragment-id \cite{TODO}
messaging as well as the \js|postMessage| API \cite{TODO}.



\subsection {Cross-origin resource sharing} 



% Local Variables:
% TeX-master: "main.ltx"
% TeX-command-default: "Make"
% End:

