\section{Background}
\label{sec:background}



\subsection {Same origin policy}

The same origin policy is an important security concept for a number of
browser-side programming languages~\cite{TODO}. This policy aims to maintain a strict
separation between content provided by unrelated sites. It allows scripts
running on pages originating from the same web site\footnote{An origin is 
defined as a combination of scheme, hostname, and port number} to obtain data
by \js|XMLHttpRequests|. Additionally, scripts from the same origin are allowed to
access each other's DOM with no specific restrictions, but it prevents access to
DOM on scripts from different sites. 

Despite SOP, scripts can still achieved cross-origin communication in several
manners. Modern browsers impose no restrictions on how URLs are used. Scripts
are thus allowed to embedded images, scripts, styles and frames from arbitrary
domains, where the URLs used for that purpose encode information from the
current web page. Moreover, the SOP does not forbid performing cross-origin
requests (only to observation of the responses): thus, it is trivial to write
data to a third-party website. Finally, in the browser, cross-origin windows and
frames can freely communicate bi-directionally by the use of fragment-id \cite{TODO}
messaging as well as the \js|postMessage| API \cite{TODO}.

The SOP is sometimes too liberal and to conservative some other times. In this
light, browsers adopt different mechanisms to either restricted or relaxed it.
We briefly describe them below. 

\subsection {Content security policy} 

Content security policy  (CSP) \cite{TODO} helps to significantly reduce
the risk (neglected by SOP) of freely handling URLs. CSP essentially describes
a, on a page-by-page basis, which the origins from where different resources
(scripts, images, etc.) may be safely loaded, i.e., it works as a simple
white-list policy.  Web pages rely on server-support to insert CSP headers in
\js|XMLHttpRequest|s responses. For instance, a page hosted in \js|a.com| 
which supplies the CSP header with the directives 
%
\js|default-src: ’self’; img-src: b.com| 
%
restricts the page to load images from \js|a.com| and \js|b.com|.



\subsection{Cross-origin resource sharing } 


There are situations where it is desirable from scripts at different domains to
communicate. With this in mind, cross-origin resource sharing (CORS) relaxes SOP
by implementing a new browser-server protocol through HTTP headers.  Such
headers specify from which domains scripts are allowed to fetch data 
using \js|XMLHttpRequest|s. Moreover, in the browser, scripts loaded from
those cross-origins have unrestricted access to each other's DOM
trees. Unfortunately, any technique to subvert SOP works equally well with CORS 
(e.g. fragmented-id, postMessages, etc.).


% Local Variables:
% TeX-master: "main.ltx"
% TeX-command-default: "Make"
% End:

