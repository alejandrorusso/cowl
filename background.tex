\section{Background}
\label{sec:background}



\subsection {Same origin policy}

The same origin policy is an important security concept for a number of
browser-side programming languages~\cite{TODO}. This policy aims to maintain a strict
separation between content provided by unrelated sites. It allows scripts
running on pages originating from the same web site\footnote{An origin is 
defined as a combination of scheme, hostname, and port number} to obtain data
by \js|XMLHttpRequests|. Additionally, scripts from the same origin are allowed to
access each other's DOM with no specific restrictions, but it prevents access to
DOM on scripts from different sites. 

Despite SOP, scripts can still achieved cross-origin communication in several
manners. Modern browsers impose no restrictions on how URLs are used. Scripts
are thus allowed to embedded images, scripts, styles and frames from arbitrary
domains, where the URLs used for that purpose encode information from the
current web page. Moreover, the SOP does not forbid performing cross-origin
requests (only to observation of the responses): thus, it is trivial to write
data to a third-party website. Finally, in the browser, cross-origin windows and
frames can freely communicate bi-directionally by the use of fragment-id \cite{TODO}
messaging as well as the \js|postMessage| API \cite{TODO}.

The SOP is sometimes too liberal and to conservative some other times. In this
light, browsers adopt different mechanisms to either restricted or relaxed it.
We briefly describe them below. 

\subsection {Content security policy} 

Content security policy  (CSP) \cite{TODO} helps to significantly reduce
the risk (neglected by SOP) of freely handling URLs. CSP essentially describes
a, on a page-by-page basis, which the origins from where different resources
(scripts, images, etc.) may be safely loaded, i.e., it works as a simple
white-list policy.  Web pages rely on server-support to insert CSP headers in
\js|XMLHttpRequest|s responses. For instance, a page hosted in \js|a.com| 
which supplies the CSP header with the directives 
%
\js|default-src: ’self’; img-src: b.com| 
%
restricts the page to load images from \js|a.com| and \js|b.com|.



\subsection{Cross-origin resource sharing } 


There are situations where it is desirable from scripts at different domains to
communicate. With this in mind, cross-origin resource sharing (CORS) relaxes SOP
by implementing a new browser-server protocol through HTTP headers.  Such
headers specify from which domains scripts are allowed to fetch data 
using \js|XMLHttpRequest|s. Moreover, in the browser, scripts loaded from
those cross-origins have unrestricted access to each other's DOM
trees. Unfortunately, any technique to subvert SOP works equally well with CORS 
(e.g. fragmented-id, postMessages, etc.).


\subsection{Lattice-based security}

%% The idea of principals
\paragraph{Principals and labels}
\sys~ specifies policies in terms of origins, i.e., principals, providing web content. \Red{check
  with Deian, is it origin (port, scheme, etc) or just domain?} 
These origins are identified by URL, e.g., 
\js|http://maps.google.com:80/| denotes a well-known map provider principal. 

Conceptually, \sys~ describes security policies by associating labels to every
piece of data. Labels express restrictions on information propagation according
to the interests of multiple web context providers. Among different
information-flow label systems~\cite{CSF2013}, we adopt the use of disjunction
category (DC) labels~\cite{TODO} due to its simple semantics. 
In this setting, DC labels are conjunctive normal form of
origins. As an example, the label
\js|Label("https://bank.ch").and("https://amazon.com")| attached to a piece
of data indicates that it might contain sensitive information from both origins;
therefore it
should not be propagated to less sensitive entities, e.g.,
\js|Label("https://amazon.com")|---otherwise, bank data could be leaked into
\js|amazon.com|. 
%% Or?

\Red{perhaps subsumes is better?}
Based on logic implication, DC labels can be ordered by their degree of
sensitivity in order to form a lattice. Considering confidentiality
(integrity proceeds in a dual manner), $l_1$ is less or equally sensitive than
$l_2$, written $l_1 \sqsubseteq l_2$, iff $l_2 \Rightarrow l_1$ when interpreting
origins as boolean variables. Alternatively, we say that $l_2$ subsumes $l_1$. 

\sys~ support unforgeable objects denoting origins called privileges. These
objects are used to assert the authority of principals, i.e., whatever scripts
running with the privilege denoting \js|https://amazon.com| is treated like
coming from that origin even if it did not. Based on the privileges in scope,
scripts are capable to restrictively declassify sensitive information.  As an
example, a piece of code running with a privilege denoting
\js|"https://amazon.com"| can relabel a piece of data with
\js|Label("https://bank.ch").and("https://amazon.com")| to
\js|Label("https://bank.ch")|; and thus effectively downgrading information to
the bank. Exercising privileges is captured by a more permissive pre-order
relationship $\sqsubseteq_p$, where $l_1 \sqsubseteq_p l_2$ provided that $l_2
\land p \Rightarrow l_1$. Observe how the privileges are used on the hypothesis
of the implication. Consequently, $l_2$ does not need to include the origins in
$p$ in order to make the implication hold. If $l_2$ is formed by removing the
origins from $p$ and appearing in $l_1$, it holds that $l_1 \sqsubseteq_p l_2$. 
 

%% Formal definition 


% Privileges
% Then, the API API for DC labels is described in Figure \ref{}.   
%% Explain the API





% Local Variables:
% TeX-master: "main.ltx"
% TeX-command-default: "Make"
% End:

